\section{Creating Adventures}\label{sec:gm-adventures}
\begin{multicols}{2}
Most adventures are designed around a structure that looks like this: You start a scene, linearly follow it for over a set number of scenes that occur in a predetermined sequence, and it comes to an end. Occasionally you might see someone want to vary playing a pseudo-option, where characters can go through 2 or 3 alternative scenes before returning to the main script. You're still looking at an essentially linear path, though. Although the exact form of this linear path may vary depending on the adventure in question, ultimately, this form of design is the linear approach: A happens, then B does, and then C happens.

The main advantage of the linear approach is its simplicity. It is easy to understand and easy to control. When you are preparing the adventure, it feels like assembling a task list or telling the plot of a short story. While you are running the adventure, you always know exactly where you are and exactly where you should be going. But this approach has two major drawbacks:

Firstly, lack flexibility. Each arrow in the designed flow chart is a bottleneck: If the players do not follow that arrow (because they do not want to or because they do not realize what they should do), then the adventure will grind to a halt. The risk of this painful derailment (or the need to railroad your players) can be mitigated, but if the players realize you are guiding the actions of their characters, they can feel like you're playing in their place. (This, ironically, can cause them to rebel against your best plans.) Second, because it lacks flexibility, linear approach is the enemy of players’ agency. In order to finish the adventure, the characters must follow the arrows. Choices that do not follow the arrows will break the game. For a game that values player empowerment starting with character creation, to create linear adventures goes against the rest of the system.

Of course, it is easier said than done, as one of the reasons that people prefer the linear approach is that it provides a significant structure of prepared situations: It says exactly where to go and how to do to get there. Without this kind of structure, it is very easy for a game session to derail. Certainly, it is not impossible to simply let the players loose, improvising everything and end up somewhere interesting. Similarly, it is quite possible to get in a car, driving aimlessly for a few hours, and have a nice exciting trip to tell. But is often helpful to have a territory map.

This line of thought, however, often leads a dilemma. The logic is like this: (1) I want my players to have meaningful choices. (2) I need to have a framework for my adventure. (3) Therefore, I need to prepare for each choice players can do. The result is an exponential expansion adventure path. The choice of “A” leads to two paths, A1 and A2, which have the choice “B” which takes four paths, A1B1, A1B2, A2B1, A2B2, which have the “C” choice that takes eight paths, and so on. The problem with this project should be evident: You are preparing several times more material to provide the same amount of playing time. And most of the material you are preparing will never be seen by the players.

In a way, of course, this is an extreme example. You can simplify your task by collapsing of some of these options, like having A2B1 and A1B2 choices lead to the same place. But, yet, you are designing several scenes in order to provide only three real game scenes. You’re still specifically designing a material you know that will never be used. And, if you think about it, in fact this example is not as extreme as well: The initial example assumes that there are only two possible choices at a given moment. In fact, it is quite possible that there are three or four or even more choices. And each additional option adds a whole new series of contingencies you need to account for.

Ultimately, this kind of adventure “Choose your own path” is a dead alley: No matter how much you try to predict ahead of time, your players will still find options you never considered --- forcing you back to the position of artificially limiting the choices to keep intact your preparation or leaving you with just the same problem you were trying to solve in the first place: to avoid excessive improvisation. And even though this is not true, you're still burdening yourself with a preparation process filled with unnecessary work.

The solution to this problem is node-based design. And the root of this solution lies in reversing the Three Clues Rule. This rule states: “\textbf{For any conclusion you want players to have, include at least three clues.}” The theory underlying this rule is: if you present three different options, it provides enough redundancy to create a robust scenario. Even though the characters lose the first clue and misinterpret the second, the third clue provides an ultimate safety net to keep the scenario working. This logic, however, also leads to the reversal of the Three Clues Rule: “\textbf{If the characters have access to three clues, they will get at least to a conclusion.}”

In other words, if you need it players reach three conclusions (A, B, and C) and the characters have access to three clues (each of which theoretically allow them to achieve these conclusions), then it is very likely that they will in fact achieve at least one of these conclusions. Understanding this reversal of the Three Clues Rule allows us to embrace all the flexibility of node-based design.

Imagine an adventure that begins in a node (or scene), containing three clues- one pointing to the node A, one for node B, and one for the node C. Using the inverse of the Three Clues Rule, you know that players will be able to conclude that they need to go to at least one of these nodes. Suppose they go to the node A. There, there contains two additional clues --- one pointing to the node B and another that points to the node C. At this point, the characters have access to five different tracks. One of them led them successfully to the node A and can now be discarded. But that leaves them with four clues (two pointing to node B and two pointing to node C), and, because the inverse of the Three Clues Rule, once again we see that they have more than enough information to proceed further.

Let's assume now go to the node C. Here they find clues to nodes A and B. They now have access to a total of seven clues. Four of these clues now point to nodes that have been visited, but the three last clues point to the Node B. The Three Clues Rule shows that they now have access to enough information to complete the adventure.

Although there was no change between linear adventure and the node-based adventure apart from the event order, the simple fact that the players are sitting in the driver's seat is important. Even if the choice has no lasting impact on the final conclusion of “good guys win, bad guys lose” the fact that the players were the ones who decided how the good guys did win is important.

And this becomes even more relevant when Traits come into play. Traits should be placed as key points when creating nodes, as they are exactly the characters’ main objectives, and, most importantly, point the stories that your players chose to play! Thus, the clues that lead from one node to another should talk to the characters through their goals.
\end{multicols}

\section{Character Development and Rewards}\label{sec:gm-chardev}
\subsection{Experience}\label{subsec:gm-experience}
\begin{multicols}{2}
The game speed depends on the will of the group about the campaign’s duration. FFRPG characters start the game with 200 experience and can reach a maximum of 25,120 experience points, starting the game between levels \ordinalnum{4} and \ordinalnum{8} and reaching the \ordinalnum{100} level with around 25,000 XP\@. However, it is recommended that the game ends around level 65, or about 11,000 XP for each character. At this level, the characters are already powerful and experienced enough to overcome most of the possible challenges without too many problems.

With these numbers in mind, the GM can plan the campaign’s progress according to the time available to play it. A campaign that lasts one year, playing once per week, for example, in “normal speed” may give, on average, about 200 XP per session (assuming two things: first, 200 XP times 52 weeks equals 10,400 XP at the end of one year, and second, the sessions of the campaign’s start will give less than 200 XP and the campaign’s final sessions will give more than 200 XP). This speed can and should be adjusted for the campaign’s estimated time, the size of the story being told and frequency of play.

At each node on your
adventure, identify which Traits
are covered by that node. A node
that involves a mother’s
desperate pleas to save her son
from the clutches of a ferocious

monster can, for example, involve the Traits Reputation, People’s Hero and Monster Hunter. Let's say that the GM has stipulated a total of 160 XP for that node; it may give 80 XP if the characters save the child, reducing to only 40 if it is wounded in the process; give a bonus 0--20 XP based on characters’ actions and their ability to become famous with the situation in the eyes of the other villagers; and give also 20 XP if the characters manage to kill the beast, rising to 60 XP if they deduct, based on the given clues, the reason for the attack and decide to only scare away the beast in order to follow her into the villain's lair.

Thus, characters can earn between 0 and 160 XP for this node, based on their actions and their Traits. When deciding their Traits, players pass a clear message about what kind of situations they want to face. Use this to your advantage to create exciting adventures for your group.

Between sessions, players can spend experience points to increase their Stats, gaining levels and Skills. Alternatively, the GM can decide the experience points may only be spent  between adventures or  in “training periods” defined by the GM\@. 

\begin{boco}
Optional Rule: \textbf{Group Experience}

Some groups may find it too taxing on the GM to assign individual experience. Maybe they feel it is hard to do, or maybe they feel that it is unjust that the characters may earn different experience values.

If you prefer, instead of granting experience based on each character's Traits, you may give them based on the group's Traits. This way, all characters will earn the same experience values at each session, based on the group's actions. Individual Traits should be recorded for Destiny purposes, but they won't give different experience totals for each character.
\end{boco}

\subsection{Traits and Quirks}\label{subsec:gm-traits}
During the course of a campaign, the characters will evolve in a different way aside from the mere accumulation of experience. Fundamental changes in a character’s goals, personality and characteristics may imply changes to its Traits and Quirks. For example, a character may see the problems caused by his Arrogant personality and decide to become Empathic, or maybe the death of a Protégé when an NPC stick his sword in her back can turn the story into a tale of revenge against this new Nemesis.

Anyway, whenever a character has a significant reason to change their way of being, independent if that reason is internal or external, it can, between sessions, change his Quirks. To do so, the player declares this to the Game Master and changes it. In the case of the Traits, whenever one of the three Traits chosen by the Group change by players consensus or the imposition of narrative (the heroes may beat a Nemesis, redirecting their goals to something else, like fetching Relics of the Past), all players, regardless of their own character’s Traits, may change them, to reflect the change of group goals.

\subsection{Creating Experienced Characters}\label{subsec:gm-expchars}
If, for any reason, you wish to create experienced characters, simply follow the normal character creation model and assign more than 200 XP\@. This is useful when a character enters the game after its start, or when the group decides to start the game with a greater power level than starting characters. The GM must decide how much XP the character should have, and the player assign it to its Stats as normal. Also, instead of having 200 Gil for starting equipment, the character may acquire equipment suited for its level.

\begin{mog}
In some groups, you'll find a common kind of player: the powergamer. These players focus on creating the most optimized character, ready to tackle any obstacle the GM can provide. Many players frown from them, as they'll search for every loophole and rules exception to try to create the “unbeatable” character. Most also use minmaxing techniques: they overspecialize to exploit high-level content while “dumping” or ignoring other areas of the character. \pc%

Most RPGs do have countermeasures to stop minmaxing and powergaming. This game\ldots{}\ well, it has none. No explicit rule, at least. The game engine is designed to punish minmaxers. Got a character that overspecializes in Earth? Throw him an opponent with \taction{Parry} and high Air value and he'll be a sitting duck. Enormous HP and defenses? Throw some \tstatus{Gravity} effects and he's toast. \pc%

However, the biggest deterrent to overspecialization is the Stats' XP cost. You don't need to throw curveballs: just let the minmaxer do his magic\ldots{}\ and soon he'll be in a lower level than his balanced friends, and will be weaker than his allies, except in situations suited for his specialty.
\end{mog}
\end{multicols}

\section{Antagonists}\label{sec:gm-antagonists}
\begin{multicols}{2}
An important part of creating interesting challenges for players is to create interesting antagonists. Just as there are no shadows without light, is the quality of antagonists that will magnify the story created for any work of fiction, be it a book, a movie, a video game or an RPG campaign. Thus, this chapter is intended to help Game Masters to craft balanced and challenging antagonists for their players. As any crafting, this process has equal parts art and science. As any crafting, the creation process should take into consideration the target audience: a monster that is challenging to a group can be simply deadly to another, or even a walk in the park for a third group. As any crafting, even when following established recipes, know that the you can always fine-tune the details. To create antagonists, use the following script:
\begin{enumerate}
    \item \textbf{Choose the Types}
    \item \textbf{Pick a Class}
    \item \textbf{Decide Stat Levels}
    \item \textbf{Set HP, MP, ARM, and MARM}
    \item \textbf{Include actions and finishing touches}
\end{enumerate}

\subsection{Type}\label{subsec:gm-types}
Each antagonist must have at least one type. This type is important for Abilities and attacks with effects like (monster) Killer and (monster) Destroyer. In addition, they often possess characteristics and skills in common, although they are not mandatory. A monster may have more than one type.

\tmtypeaberr{}: A catch-all type that covers the monsters that can’t be categorized in any of the other types. Due to its breadth, there are no characteristics and abilities that define this type.

\tmtypeaqua{}: Creatures that live in water, typically, although many can act on dry land. Usually have \tstatus{Lightning Vulnerable} and \tstatus{Water Resist} or \tstatus{Water Immune}.

\tmtypeconstr{}: Golems and machines, made of stone, wood, metal or other materials, animated by magic or technology. Usually they have resistance or immunity to several conditions, such as the \tstatus{Mental}, \tstatus{Toxic} and \tstatus{Transform} types. Depending on the material, they may have elemental vulnerabilities and immunities.

\tmtypedemon{}: Supernatural antagonists serving the cause of evil, usually with great magical powers. Usually have \tstatus{Light Vulnerable} and \tstatus{Shadow Resist} or \tstatus{Shadow Immune}.

\tmtypedrgn{}: Monstrous reptiles, ranging from snake-like creatures to large winged lizards. Drakes range from small power to legendary creatures of similar power to gods. Due to its breadth, there are no characteristics and abilities that define this type.

\tmtypeelem{}: Creatures with a deep relationship with a specific element. Usually have immunity or even the absorption to that element.

\tmtypebeast{}: The “natural” world inhabitants, including common animals, monstrous animals and other wildlife, even those distorted by magic. Often have \tstatus{Ice Vulnerable}.

\tmtypehuman{}: Creatures of with biology and proportions similar to humans. Often have \tstatus{Shadow Vulnerable} and \tstatus{Bio Vulnerable}.

\tmtypeundead{}: Unliving creatures reanimated by supernatural circumstances or ghosts that haunt the living. Usually they have \tstatus{Fire Vulnerable} and \tstatus{Zombie}.

\subsection{Class}\label{subsec:gm-class}
Besides the type, you must set the antagonist’ Class. Each antagonist has a class that defines his level of power. \tmobmini{}s are the cannon fodder and represent the low power antagonists that make up most of the opponents.

\tmobcomm{}\ antagonists are a step above the minions, being as powerful as the player characters, representing a greater threat.

\tmobleet{}\ antagonists are unique creatures, veterans who can be more powerful than the player characters, requiring teamwork to be defeated.

Finally, \tmobboss{}\ antagonists are the most powerful, often representing the climax of not only adventures, but campaign arcs.

\subsection{Antagonist Stats}\label{subsec:gm-stats}
Antagonists have the same four Stats as player characters: Earth, Air, Fire and Water. Unlike the player characters, they do not need to use XP to increase their Stats. When creating your antagonist, decide the Stat values according to the Group’s power and the proposed challenge.

Regardless of the antagonist’ class, consider their Stats balanced if they are between 40 points lower and 40 points higher than the player characters’ Stats. Antagonists have no Traits, Quirks, Equipment, Skills or Destiny Points. Their character level is the sum of its Stat levels.

\subsection{Combat Values}\label{subsec:gm-comvalues}
The HP, MP, armor and magic armor values depends on the level and Category, according to the chart below.

\begin{center}
    %\rowcolors{1}{zebragray}{}
    \begin{tabular}{*{4}{M@{-}O}}
        \toprule
        \rowcolor{zebragray} \multicolumn{2}{c}{Level} & \multicolumn{2}{c}{HP} & \multicolumn{2}{c}{MP} & \multicolumn{2}{c}{(M)ARM} \\ \midrule % chktex 36
        1 & 9 & 32 & 64 & 10 & 50 & 2 & 12 \\
        10 & 19 & 80 & 148 & 16 & 64 & 4 & 24 \\
        19 & 27 & 216 & 288 & 24 & 148 & 13 & 48 \\
        28 & 36 & 388 & 468 & 62 & 288 & 24 & 84 \\
        37 & 45 & 604 & 704 & 80 & 468 & 37 & 120 \\
        46 & 54 & 892 & 1000 & 100 & 704 & 62 & 184 \\
        55 & 63 & 1208 & 1332 & 150 & 1000 & 75 & 245 \\
        64 & 74 & 1984 & 2260 & 250 & 1332 & 123 & 368 \\
        75 & 84 & 2320 & 2484 & 400 & 2260 & 144 & 452 \\
        85 & 100 & 2544 & 2820 & 500 & 2484 & 158 & 555 \\ \bottomrule
    \end{tabular}
\end{center}

\tmobmini{}s should have half table’s HP and roll two initiative dice at the start of each round. \tmobcomm{} antagonists have table’s HP and roll three initiative dice at the start of each round. \tmobleet{}s should have double or triple HP and up to twice table’s MP and roll three or four initiative dice in each round. \tmobboss{}es must have between five and six times HP and up to five times table’s MP and roll four or five initiative dice at the start of each round.

The ARM and MARM values apply to all Classes. When deciding the ARM and MARM values, avoid creating creatures without weaknesses. Antagonists with high armor usually have low magic Armor, and vice versa.

\subsection{Abilities and Actions}\label{subsec:gm-abilities}
All antagonists should have at least a basic attack, which can be physical or magical. Choose an Offensive Stat and Defensive Stat for the attack. The table below shows the adequate damage multiplier for each antagonist’s level.

\begin{center}
    \begin{tabular}{*{4}{M@{-}O}}
        \toprule
        \rowcolor{zebragray} \multicolumn{2}{c}{Level} & \multicolumn{2}{c}{Damage} & \multicolumn{2}{c}{Level} & \multicolumn{2}{c}{Damage} \\ \midrule
        1 & 9 & 2 & 4 & 46 & 54 & 13 & 17 \\
        10 & 18 & 3 & 7 & 55 & 63 & 14 & 20 \\
        19 & 27 & 5 & 9 & 64 & 74 & 16 & 24 \\
        28 & 36 & 8 & 12 & 75 & 84 & 18 & 26 \\
        37 & 45 & 10 & 14 & 85 & 100 & 20 & 35 \\ \bottomrule
    \end{tabular}
\end{center}

The antagonists should have at least a basic attack, which can be physical or magical. Choose an Offensive Stat and a Defensive Stat for the attack and its damage. For example, an attack of an antagonist between level 10 and level 18 should cause between 3x and 7x Stat level damage.

\textbf{Earth}: Attacks with Earth as Offensive Stat depend on user’s strength and muscular power. They are usually linked to brute force and physicality. Attacks that can be prevented by the defender’s physical strength, health and muscle power of should use Earth as Defensive Stat.

\textbf{Air}: Attacks with Air as Offensive Stat depend on the user’s skill and speed. They are usually linked to the precision and finesse. Attacks that can be avoided by the defender’s agility and the reflexes of should use Air as Defensive Stat.

\textbf{Fire}: Attacks with Fire as Offensive Stat depend on the user’s intelligence and magical ability. They are usually linked to cunning and the supernatural. Attacks that can be avoided by the defender’s insight should use Fire as Defensive Stat. It is a Stat rarely used defensively.

\textbf{Water}: Attacks with Water as Offensive Stat depend on the user’s charisma and willpower. They are usually linked to luck and force of personality. Attacks that can be prevented by the defender’s mental strength and magical defense of should use Water as Defensive Stat. It is a Stat rarely used offensively.

Antagonists can use Spells and both Main and Secondary Jobs’ Abilities, causing effects at players’ similar levels, although they still must follow the damage table. For example, the Fire Spell is suitable for antagonists of \ordinalnum{10} to \ordinalnum{27} level, as it deals 5x Fire level damage. In the case of actions that depend on the equipped weapon, like \taction{Charge} or \taction{Black Sky}, simply decide the damage based on level limits.

Finally, \tmobleet\ or \tmobboss\ antagonists may have actions as if they had up to 9 levels higher than their actual level. These abilities are considered special attacks and should be used very sparingly. Thus, a \ordinalnum{40} level Boss could have an attack dealing 17x Stat level damage.

In addition to the actions and reactions that are appropriate to the antagonist, feel free to grant the conditions that make sense for the creature, like \tstatus{Flight}, \tstatus{Float}, \tstatus{Zombie}, \tstatus{Elemental Vulnerable} or \tstatus{Status Vulnerable}, \tstatus{Elemental Resist} or \tstatus{Status Resist}, \tstatus{Elemental Immunity} or \tstatus{Status Immunity}, \tstatus{Elemental Absorb}, among others.

If you’re new to the FFRPG \ordinalnum{4} Edition, feel free to experiment with your antagonists and test the power level of your group. Different strategies and stat combinations can be either deadly or a push-over, based on the group’s capabilities. When in doubt, put your enemies in the weaker side and increase their power slowly, until you’re comfortable enough with the system.

\subsection{Final Details}\label{gm-findet}
With the antagonist’s mechanics done, be sure to flesh out the details that give life to roleplay. These details can be as specific or as general as necessary; an unimportant \tmobmini\ may simply have a general description, but an \tmobleet\ who will star an entire adventure should be detailed enough for your players love to hate their enemy.

What's its name? What’s its physical description? What’s its personality? Does it have habits, peculiarities, catchphrases? What are its goals, fears, anxieties? What’s its natural habitat? How it relates to the adventure and the other antagonists? These are only some of the questions that you, as a GM, may ask to give depth to the antagonist.
\end{multicols}
\vspace{\stretch{1}}
\begin{center}
  \adjincludegraphics[width=0.9\textwidth]{block-combat}
\end{center}
\vspace{\stretch{1}}