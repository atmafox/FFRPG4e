\section{Play Basics}\label{sec:basics}
\begin{multicols}{2} % Fixme: What the heck is wrong with spacing here
This PDF contains the entirely of the rules
you'll need to know. Like most pen-and-paper
RPGs, you'll also need dice – in this case, ten-sided
dice. Most rolls in this system are referred as a
“d100” roll. To do that, roll two ten-sided dice,
usually in two different colors or by rolling one
after another. One die will represent the unit digit,
and the other die will represent ten's digit. Name
what die will be each digit before you roll! A result
of “00” is read as 100. You’ll usually want two per
player but having extras is a good idea.

\subsection{Setting}\label{subsec:setting}
The Final Fantasy series has spanned
dozens of worlds – from Ivalice, the multiracial
land of mercenaries and adventure, to Cocoon, the
artificial planet nurtured by psychopathic ancient
constructs – and rarely do any two of these worlds
operate under the same laws. Some feature magic
that slowly chips away at the minds of those who
use it, and in others death is little more than a
minor inconvenience.

As you can imagine, this means that creating
an universal set of rules, one that covers
everything from the entire Final Fantasy series, is
simply impossible – and ignoring this fact would
be irresponsible and downright chaotic at best.
Details about each specific game setting may be
found on the Worldbooks series, each describing
one specific world. At the time of this book, the
Worldbooks for Final Fantasy IV, VI, and Tactics are
already published, and many more are coming.

That said, this book was designed to give you
and your friends a chance to explore your own
stories, with your own heroes, and very often, in
your own unique Final Fantasy world. Whether
your campaign will be set a familiar place or
whether it will be set in a post-apocalyptic city
where the last memories of the deceased are
immortalized as whispering magical crystals\ldots{}\ well, that’s all up to you. Are the heroes tied
together by ancient prophecy or are they
childhood friends? Are they a crew of gentlemen
airship thieves or forced into an uneasy alliance by
the outbreak of war? This is a chance to let your
creativity shine.

\subsection{The Game Master}\label{subsec:gm}
In the video games, Final Fantasy may be
accessed with a cartridge, CD-ROM or DVD. In
FFRPG, however, it is the Game Master (GM) who
unspools the epic saga, acting as both referee and
storyteller. As a storyteller it is their responsibility
to create the quests and storylines the players
become embroiled in, take on the roles of Non-
Player Characters (NPCs), the people and
monsters the heroes encounter in their travels,
and act as the players' eyes and ears within the
game, describing the scenery and situations. As a
referee, the GM enforces the rules, set outs the
challenges, and keeps the players on task to ensure
each session runs as smoothly as possible. It might
seem daunting to tackle these challenges with the
patience and dedication they deserve, but it is very
rewarding.

\subsection{The Players}\label{subsec:players}
Players in the FFRPG step into the shoes of a
character with a unique background, personality,
skills and powers. These protagonists are known
as the Player Characters (PCs), and ultimately
shape the story by virtues of their actions and
decisions. There are some crucial differences
between videogame and tabletop play, however;
each player generally only control one character,
rather than an entire party. As result, most
adventures will see several players cooperating
with each other under the GM's guidance, working
along to create the story. This game is not a
competitive exercise: the players work together to
overcome the challenges they create and/or the
GM presents, but the collective goal is to have fun
together, not to “win”. As much as the PC and some
NPC may be enemies, players and the GM have the
same purpose: to collectively have fun while
creating a memorable story.

\begin{mog}
Kupo! I'm Mog, the moogle, and I'll be your guide
through this book. I'll be presenting you some examples,
hints, and other assorted help so you can understand
more easily the terms and how the game works. I hope
you have fun playing this edition of the FFRPG!
\end{mog}

\begin{boco}
Quark! I'm Boco, the chocobo. During the
book, I'll bring mechanical advice and optional
rules. There are plenty of ways to use this system,
and you can change it to better suit your playstyle
and your playgroup. So, count on me to bring
alternate ways to use this ruleset to your liking!
\end{boco}
\end{multicols}

\section{Design Principles and Decisions}\label{sec:design}
\begin{multicols}{2}
The game's core concept is ``nostalgia''. As
such, the game must be flexible and customizable
enough, so each group can emphasize the aspects
they find most relevant to the gaming experience.
Final Fantasy have over 25 years of history and
different gamers had lots of different experiences.

The gaming unit is the group. Use
mechanical rules for the group of adventurers as
design decision. This ``group'' should consider not
only the players ‘opinion, but also the Game
Master's.
\subsection{Avoid Downtime}\label{subsec:downtime}
Downtime is the time that the player spends
``not-playing'' during the game. It is basically
caused by the fact that not all players can
effectively act at the same time, due to the Game
Master’s inability to hear everyone at the same
time. During the gaming experience, all players
must be involved in the action as much as possible.
\subsection{Quick Production}\label{subsec:production}
Translating Final Fantasy to tabletop is a
huge endeavor. It is best to deliver a small product
that can be played with few options than try to
describe all the possible options before publishing
the game. The game should be modular, with space
for ``expansions'', to allow quicker production. The
specific rules for specific game settings should be
published into each game module, and the game
should be easily “moddable”. Also, this means this
game can and will be subject to errata,
supplements, and needs to be able to evolve with
time.
\subsection{Create Two Games}\label{subsec:twogames}
Yes. The idea is to two games. If you stop to
coldly analyze what were the JRPGs of the golden
era (8, 16 and 32 bit), you will see that basically
the ``RPG'' is a strategy game and an exploration
game (I'm talking of pure RPGs, not Action RPGs or
Tactical RPGs). The moment of transition between
the two games is the moment of combat. Thus, to
emulate this spirit, I am divorcing the combat
system from the ``non-combat'' system to clearly
characterize the two moments. Actions out of
combat will have another mechanics and will
function differently from actions in combat.
\subsection{Tabletop, But Not Only Tabletop}\label{subsec:tabletop}
This is a tabletop game and should be
written as such. The rules should be as tabletop-
friendly as possible and should be able to play with
only pencils, paper and dice. However, RPG
evolved with its medium over the time. The rules
must be able to work in other environments, like
computer assisted, chat-based games and the
slower Play-by-Post games.

\end{multicols}