\label{ch:preface}
\markboth{Preface}{}

\begin{center}
    \adjincludegraphics[width=0.9\textwidth]{block-aeris}
\end{center}
\section*{Forward}
\label{sec:forward}
\begin{multicols}{2}
Before presenting this product, I would like to introduce myself. I am an RPG player and Game Master since 1994, and one of the people who were part of the Returners group of since the creation of FFRPG’s first edition. I followed and helped to carry out the playtest of all ``official'' editions of the game, through the internet relay chat (IRC) campaigns or playing offline. The group stayed together until the end of second edition’s development, around July 2001.

When the development of the famous third edition started, there began to be disagreements among the team, leading to the creation of the first FFRPG ``spin-off'', called ZODIAC, created by S Ferguson, which, from 2001 until today (the last version is dated 2013), has gone through three editions. The FFRPG (or Returners' FFRPG, as it became known to differentiate the ZODIAC FFRPG) took 8 years to complete the development of its 3rd edition, being officially launched in 2009. But even after such a long development time, it was released with various problems that generated other versions.

There were the works of Fernanda Parker in Brazil, who translated the game to Portuguese; the SeeD group, formed by many of the original members of the Returners, but already without the leadership of Samuel Banner, who was project leader at the time of the 3rd edition, created another version, the FFRPG SeeD, more focused on online gaming via IRC: according to the authors, the game is ``impossible'' to be played on a real table without computer assistance; Alan Wiling and his team created the FFRPG d20, based on Pathfinder; Scott Tengelin and his team also started another version, which was finalized by Dust, who published, in 2010, the third edition of FFd6 system. Another game published was Chikago’s Academia Bahamut, a Brazilian cross between ZODIAC and the original 3rd edition.

Thus, when creating this game, I try to stand on the shoulders of giants. All creators and games mentioned above were sources of inspiration when I build this version. Still, I do not consider myself writing the final or definitive version of FFRPG; as well as electronic games, pen and paper RPG continues to evolve and incorporate different concepts and reach different audiences. I hope that this work is just the kickoff to the fifth edition, sixth edition, and many others, created by fans of this ageless masterpiece: Final Fantasy.

This game is not affiliated, owned or subject to Square Enix or any of their companies in any way. Much of the material described here is owned by Square Enix and its various contributors as Yoshitaka Amano. This work is licensed under the Creative Commons Attribution-NonCommercial 4.0 International License. To view a copy of this license, visit\\
\url{http://creativecommons.org/licenses/by-nc/4.0/}\\
or send a letter to:

\hangindent=\parindent
Creative Commons\\
PO Box 1866\\
Mountain View, CA 94042\\
USA


- \ferrum{Bruno Carvalho}\ferrum
\end{multicols}
\newpage

\begin{multicols}{2}
\section*{Acknowledgements}
\label{sec:acknowledgements}
I would like to say thank you to the play testers who helped shape this game: Felipe ``Hitoshura'' Furtado, Gabriel ``Paladin'' Sasso, Anderson ``Zada'' Tavares, Wyohara, Lucas Hoffman, Leandro ``Maromba'' Valente, Pedro Rodrigues, Rafael Sobreira, Thiago Sobreira, Pedro Rodstein, Lyaran, Gregory ``Gatts'', David Renaud and everyone else. A huge thanks also to Dust, who graciously gifted her own FFd6 images and lots of fluffy bits, and Novacat for the grammar
help. \pw

For this revised edition, I would really like to thank the guys at the /r/ffrpg subreddit, including, but not limited to /u/Box\_of\_Hats, /u/kaiten619, /u/\_FunnelCakeSoda, /u/GM\_3826, /u/YouCanTrustAnything, /u/omegafantasy, and /u/StorytellerZeke. Also, thanks to Sven and Mark for your very constructive criticism.

\section*{What is Final Fantasy, Anyway?}
\label{sec:whatff}
Final Fantasy is a series of more than thirty console RPGs and two MMORPGs. Though each story in the series is independent, there are numerous recurring themes and elements such as airships and bright yellow avians, well-known monsters and heroic save-the-world storylines.

Originally inspired by Dungeons and Dragons, the Final Fantasy series has grown to take on a flavor all its own. It has become a setting in which the fundamental well-known limits of human capability are casually ignored, where a villain’s strength can be measured by their androgyny and size of their hair, and where only a ragtag team of heroes usually under the age of thirty are competent (or incompetent) enough to make a difference. These are stories about good versus evil, twisted technology and heroic perseverance, duality, self-sacrifice, camaraderie and love, and taking on truly legendary enemies with your eight-foot sword and magical umbrella.
\end{multicols}