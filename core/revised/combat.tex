
\section{Combat}\label{sec:combat}
\begin{multicols}{2}
    Combat is the raison d’etre of most rules in this book. Jobs, spells, equipment-- nearly everything from here to the end exists to serve the combat engine. Combat occurs in a space where all combatants can reach each other without worrying about movement, with the one exception that flying characters can only be hit by other flyers or ranged attacks. Combat proceeds in a series of \newterm{rounds} until only one side remains capable of acting on the field. Each round is divided into ten \newterm{phases}, counting up from 1 to 10, in which individual actions take place. The order in which things happen is determined by initiative.

    \review{One thing to note is that I want to make it clear that we have a specific-overrides-general rule going on somewhere earlier in the book, so we don't have to keep putting "unless otherwise stated" and just trust that people aren't going to be potatos about this.}

    \subsection{Initiative}\label{subsec:init}

        Each character has an \newterm{initiative}, which defaults to 3. This is the number of d10 that you roll for the character before each round begins. These d10 are your \newterm{action dice} and each represents an action you may take in the round. The sum of the numbers of your action dice when they are first rolled is your \newterm{phase speed}, which determines when you act in each phase.

        In each phase, players may act once for each action die they have that matches the phase number, and they take their \newterm{phase turns} in order of highest phase speed first, or highest air stat if there's a tie. When it's your turn in a phase, you use all of your dice for that phase at once. 

        If an active action die is not used or delayed, it is lost at the end of the phase.

        Certain effects may grant bonus action dice, or change the numbers on existing dice. Numbers lower than the current round become the current round, and numbers greater than 10 become 10.

        \review{Wvr: Possible Rule, unsure} When such manipulation adds more action dice to the current phase, they happen at the end of the phase, after all of the existing dice have been used.

        \review{Wvr: Suggesting New Rule} Each time a character's initiative is lowered, if they have unspent action dice in the current round they must choose one to discard. Each time a character's initiative is raised, if they have used fewer non-bonus dice this round than the new initiative, they roll a new die and add the current phase to its number.

        \begin{boco}
            Optional Rule: \textbf{Team Initiative}\pc%

            To simplify the amount of data to track each phase, dispense with phase speed and tie breaking: let all of the players with dice in a phase act in whatever order they desire, and then take your actions for that phase in whatever order you desire.\pw%

            \review{Wvr: Possible} Optional Rule: \textbf{Entering a Round Late}\pc%

            At the GM's discresion, a character entering combat mid-round should roll less than their usual number of dice depending on how late they're entering, and add the current phase to each die rolled. It may simply be too hectic to enter properly until the next round begins, however. \review{Bruno: It may be easier to just as normal and drop all dice that rolls lower than the current phase.}

        \end{boco}

    \subsection{Action Types}\label{subsec:actions}

        \subsubsection{Abilities}
            \newterm{Abilities} are actions are used when it is your turn. They use up the current die that is making it your turn.
        
        \subsubsection{Reactions}
            \newterm{Reactions} are acitons that are executed to a speific \newterm{trigger}. For example, the common action \taction{Dodge} triggers when you are hit by an attack. They require a die and use up the next one you have, either a held one or the next lowest number one. They may only be used once per triggering event, and using them when triggered is not mandatory.

        \subsubsection{Interrupts}
            Once per round, you may spend a destiny point use any ability \textit{right now}, or use a reaction that has been triggered without expending a die for it. This doesn't bypass charge times for slow actions, however.

    \subsection{Action Modifiers}
        \subsubsection{Free Actions}
            \newterm{Free} actions are actions which cost no dice to use. They may have other restrictions on when they may be used, such as being a reaction or being used for free as part of another action.

        \subsubsection{Continuous Actions}
            \newterm{Continuous} actions cause an effect that lasts until the next time you can act. For example, \taction{Defend} is a continuous action that reduces incoming damage until your next turn. You can't use reactions while executing a continuous action.

        \subsubsection{Slow acitons}
            By default, actions are \newterm{quick} and \newterm{resolve} as soon as you decide to use them. \newterm{Slow} actions resolve a set number of phases after they are started; a slow(N) action used on phase 1 will resolve on phase 1+N, before any other actions on that phase.
            
            A slow action's delay is called it's \newterm{charge time}, and a character preparing a slow action is said to be \newterm{charging}. While a character is charging they may not use any other actions or reacitons, and if it becomes their turn before the charging is complete they must delay or forfiet any that turn's dice. If the charge time stretches across rounds, it still takes places the same total number of phases later.

            A slow action's charge time may not be reduced below 0, and a slow(0) action is identical to a quick action.

            Targets are chosen when an action resolves. If there are no valid targets for a slow action to act upon at that point, or if the character decides at any point to abort the action, the action is forfeited and its die is not refunded.

        \subsubsection{Held Actions}
            When it is your turn you may choose \newterm{hold} your action and wait until later in the phase or a later phase to chose what action to use it for. You can delay until any point between two other actions in the round; you should state when or what you are waiting for \review{Bruno: There's no point on stating when/what you are waiting for. If it is only fluff, "you could state" works better than "you should state".} but as your character is standing watching and ready, you can change this if needed.

            You may only hold one action at any time, further action dice will need to be used or forfeited if the action is to be held past them.

            Characters incapable of acting, whether due to charging or a debilitating status effect, will automatically hold one action and then forfeit all further ones.

            You may not hold an action past the end of a round.
                
    \subsection{Action Attributes}
        \subsubsection{Physical or Magical}
            All actions are considered either \newterm{physical} or \newterm{magical}. For attacks, this determines whether the attack is affected by the target's phsyical or magical armor values; reactions and other conditions can make distinctions between them as well.

            Most weapon-related actions take their type from the weapon used, but if an action lists physical or magical in its description that overrides the weapon. Weaponless actions default to physical.
        
        \subsubsection{Melee or Ranged}
            All actions \review{Bruno: Reactions too? This makes you unable to Dodge a flying attacker. Maybe you should make all ABILITIES melee or ranged} are also considered either \newterm{melee} or \newterm{ranged}. Ranged actions can hit floating or flying targets from the ground, melee cannot \review{Bruno: It is intended to make Floating characters immune to melee? If so, either float and flight should become a single status or some other disctinction need to be made.}. As with physical and magical, the weapon property rules unless the action is specific, defaulting to melee.

    \subsection{Non-actions}
        Not everything a character can do is described by actions. Talking and gesturing, for example, are explicitly not actions. You should attmept to respect the story behind the combat in your non-acitons, however; a round is a fairly short amount of time, so leasurely discussion is out of place here, as is sipping a cup of tea while holding a greatsword ready. These things are ultimately left to the GM's discresion.

    \subsection{Attacks and Rolls in Combat}\label{subsec:attacks}
        Aside from initiative, every roll in combat is listed as ``[Active Crystal] vs [Defending Crystal], difficulty [N]''. For example, the common action \taction{Attack} takes its crystals from the weapon used, so a bow would be ``Air vs Air, difficulty 40''. To make this check, the action-using character rolls a d100 and adds his active crystal's value --- not its level. It succeeds if the result is greater than the defender's defending crystal value plus the difficulty. So in short:

        Success = d100 + attacker active crystal value > difficulty + defender's defending crystal value

        The defender can also opt to just let the action happen to themself willingly, in which case no roll is needed. This is considered standard for healing and support abilities, but a character can attempt to resist them if they so desire.

        If the action targets a group of characters, you still only roll once. You then compare that value to the difficulty-boosted defense values of each target, effecting it if successful.

        % TODO heal-hurt words
        % TODO standard bonus name
        \review{Move this bit to damage?}
        If the action damages or restores HP or MP, add the digit in the one's place to the damage or restoration value. You may also roll a seperate d10 and add it to the value; this is much easier to set up in dice rolling software for example. Either way, this value is called the \newterm{roll bonus}.

        \review{Do we really need to seperate the bonus from multi-roll effects?}

        \subsubsection{Critical Hits}
            Only actions that explicitly list critical hit effect can score them. A critcial hit happens when the roll result should is two identical numbers (100, 99, 88, etc.) and the attack is also successful.

    \subsection{Basic Actions and Reactions}\label{subsec:basicactions}
        All characters can perform the following actions, regardless of equipment, jobs, or abilities. 

        % Actype is bugged for here right now
        \subsubsection{\taction{Attack}}
            Quick action. Attack with your equipped weapon, making a roll using its active and defensive crystals as specified for the weapon, difficulty 40. See \review{pageref} for weapon details. If unarmed, roll \eve{70}, and if successful inflict 1x Earth damage. This action causes double damage on critical hits and activates weapon effects.

        \subsubsection{\taction{Dodge}}
            Reaction. When you've been hit by a melee attack, roll \ave{50} to negate the attack's effects.
            % 40-60-80 with armors later

        \subsubsection{\taction{Defend}}
            Continuous action. Brace for incoming damage. Reduces damage taken by 33\%.

        \subsubsection{\taction{Cast}}
            Quick magical action. \review{Bruno: By this wording, spells cannot hit flying enemies. Also you can dodge spells.} If you know any spells, this is how you cast them. The spells specify their rolls and difficulties.

        \subsubsection{\taction{Draw}}
            Quick action. No roll needed. You take an item or weapon from your inventory and hold it. If you're already holding an item, it returns to your inventory.

        \subsubsection{\taction{Item}}
            Quick action, free if used before any other action in a round. No roll needed. You use a held item or replace your equipped weapon with a held one; in the latter case the previously equipped weapon is atuomatically returned to the inventory.

    \subsection{Fleeing}
        Fleeing from combat is a group decision, and as such it should be worked out between the group and GM as a whole. If a roll is called for, half or more of the party beating the \review{highest? average?} enemy air score is a starting point, but different environments may call for different crystals or be better solved with a group skill challenge. And of course, sometimes, there's nowhere to flee to.

    \begin{center}
        \adjincludegraphics[width=0.35\textwidth]{block-suplex-train}
    \end{center}

    \subsection{Damage, Types, and Armor}
        Each amount of damage is of exactly one \newterm{damage type}. Characters and creatures may be resistant or vulerable to various damage types; they have no other effect on combat, and have no bearing on the physical or magical status of an attack.

        There are three \newterm{martial} types: \telem{crush}, \telem{cut}, \telem{pierce}; these can represent most of the ways humans can harm each other with conventional weapons. There are nine \newterm{elemental} types: \telem{fire}, \telem{ice}, \telem{lightning}, \telem{air}, \telem{earth}, \telem{water}, \telem{bio}, \telem{light}, and \telem{shadow};  these can represent damage caused by certain natural effects, exotic technology, and magic.

        \newterm{Armor} is a straight reduction per instance of incoming damage: physical \newterm{armor value} is subracted if the damage is physcal and likewise for magical damage and magical armor value. These are abbreviated to \newterm{PAV} and \newterm{MAV}
        \clearpage
    \subsection{Damage Resolution}
        Whenever a character receives damage, follow this procedure:

        \begin{enumerate}
            \item The \textbf{base damage} is the level of the relevant crystal level multiplied by the weapon, spell, or action's damage multiplier.

            \item If there's a percentile \textbf{action modifier}, apply it.

            \item \textbf{Strengthen and Weaken} statuses can add or remove another 25\% damage to physical or magical actions; apply this now.

            \item \textbf{Add the roll damage bonus}.
            \item \textbf{Reduce by target's PAV or MAV}.        
            \item \textbf{Apply critical hit effects}
            \item \textbf{Apply target statuses} like \tstatus{shell} and \tstatus{protect}.
            \item \textbf{Apply elemental resistances}, vulnerability, or absorb.
            \item \textbf{Subtract damage from target health}, activating any effects that trigger as health falls past breakpoints.
        \end{enumerate}
        \review{Wvr: I'm splitting the old point 6 into several points, not sure this is the right order or if i'm missing anything}
        \review{Bruno: There is no difference when it comes to critical hits, statuses and elemental resistances/vulnerability/absorb. Since they're all multiplication/division, order of operation does not matter.}

    \subsection{Healing and Death}\label{subsec:heal}
        
        Losing health doesn't cause any automatic penalties until you drop to 0 HP, which is as low as health is allowed to go. At 0 HP a character is considered \newterm{unconscious} and cannot act until they have been revived. Normal healing effects cannot heal an unconscious character, reviving effects do nothing to conscious characters.

        After an undisturbed night's sleep, HP and MP values are reset to their maximum, even for unconscious characters.

    \begin{boco}
        Dealing damage is the most time-consuming part of combat. Some tips to help it not become a slog:

        Pre-calculate damage for your common actions. Along with listing the multipliers and such list the final damage values: \taction{!Attack}: 42 Crushing; \taction{!Jump}: 84 Crushing; and so on. When you get hit by a strengthen or weaken effect figure it into likely actions right away. Try to get this ready while other players are making their actions. 

        Try to cancel opposite modifiers, even if the numbers aren't exactly equal. A \taction{Mighty Blow} or Critical Hit against an enemy with Protect deals normal weapon damage, as does \taction{Guardbreak} on a Vulnerable enemy, and so on. Speed trumps perfect accuracy when working by hand.

        And for working by hand, remember that 50\% is one half, and 25\% is half of a half. 10\% is the number with the one's place lopped off. Round down your calculations to speed them. To find a quarter of 138, for example, halve it first (69) then halve it again (34). 150\% is one plus half; 125\% is one plus a quarter; and 75\% is one minus a quarter.

    \end{boco}



    %{\centering%
    %\adjincludegraphics[width=0.2\textwidth,center]{block-magic-whale}%
    %}

\end{multicols}

\begin{multimog}
    Later in his adventures, JBMog, now a \ordinalnum{30} level character, was in a dungeon with his friends Rob, a \ordinalnum{28} level Warrior/Rune Knight and Nyarly, a \ordinalnum{31} level Adept/Wizard. After an unfortunate failed challenge roll, they fell into a trap, activating an iron Golem guardian. \\
    \textbf{GM}: \enquote{Start of Round 1. The creature rolls 1, 3 and 6 as initiative. Roll your initiative!} \\
    \textbf{JB}: \enquote{5, 6 and 9. I use my Preemptive Strike to change the 9 to 1. My initiatives are 1, 5 and 6!} \\
    \textbf{Rob}: \enquote{3, 3, and 8. And Nyarly rolled 4, 5, and 10} \\
    \textbf{GM}: \enquote{Phase 1. You're first JB, then Golem acts.} \\
    \textbf{JB}: \enquote{I'll strike the golem with my lance. I rolled a 61, plus my Air is 148 total. It hits?} \\
    \textbf{GM}: \enquote{You try to attack him with your polearm, but the heavy armor of the golem deflects it (The attack targets Earth plus diff 40. His value is 112, so he needed to overcome 152 to hit). He ignores your attacks while he prepares his own. He starts charging a Slow axe attack against Nyarly. We begin Phase 2, and his attack is finished. I roll a 34 (plus his 112 Earth totals 145) for a total 114 damage (110 damage plus 34's singles digit).} \\
    \textbf{Nyarly}: \enquote{Yikes! I will use an interrupt action and react to use \taction{Will Shield}. I spend the 4 and 10 dice. I rolled a 69, for a total of 149. I made it? (The GM nods, as his \taction{Will Shield} have 30 difficulty and the Golem's Earth value is 112, and 149 is greater than 142.) Nyarly creates a magical shield that blocks the blow and spends 14 MP (10\% of his 143 MP).} \\
    \textbf{Rob}: \enquote{Seems no one got actions in Phase 2, so I'm acting at phase 3. Seems that his weak spot is Air, so my first action is to draw my Meteo Knuckle. My second action (Rob can act before the golem since his initiative total is 11 and the Golem's is 9) is to use \taction{Item} to stash my Old Axe and equip the gloves.} \\
    \textbf{GM}: \enquote{The Golem starts to leak out a strange gas. It attacks all characters, rusting your equipment! Whoever got Water lower than 43 is hit with the Weaken (Armor) status until the end of the next round.} (The GM rolled a 31. 31 plus the Golem's Fire value is 113, and the attack had diff 70 and targeted the group) \\
    \textbf{JB}: \enquote{I think the only one hit was Rob (Rob nods). My turn now, right? (Nyarly's initiative total is 5, while JB's is 11) I'll use \taction{Advice} on Rob. I want him to crit with that knuckle!} \\
    \textbf{Nyarly}: \enquote{So, now it is my turn. I'll spend 60 HP to unleash a \taction{Fury Brand}. Rolled 90 for a total of 181 vs Water+30. If it hits, that's 131 Fire damage and maybe Berserk.} \\
    \textbf{GM}: \enquote{Yeah, that hits. Your staff burns with your magical fire when you sacrifice your lifeforce to summon a cleansing flame. Your strike hits true, dealing 101 damage (that's 131 minus the 30 MARM), but the creature seems to be immune to your mental effect. In the golem's turn at phase 6, he delays his action.} \\
    \textbf{JB}: \enquote{I'll charge Geomancy. No actions until phase 8? Ok. Let me roll for Geomancy! Hm\ldots{}\ 75? What’s the Major effect for Underground? Ah, Cave In. I'll spend the 35 MP to increase the damage. (JB rolls a 51 and hits) That's 101 damage (100 damage plus 51's singles digit)! Take that!} \\
    \textbf{GM}: \enquote{Rocks fall, and the Golem receives 71 damage from the cave in, but he's still rocking!} \\
    \textbf{Rob}: \enquote{Phase 8? At last! I'll attack him\ldots{}\ Oh, only 11\ldots{}\ I miss\ldots{}\ No, wait, I crit due to JB!\@{}Thanks! I'll do-} \\
    \textbf{GM}: \enquote{Actually the Golem reacts with his delayed action. He rolls a 17 to block and\ldots{}\ (checks Rob's Earth value of 111) fails. Your critical does only 19 damage due to the golem's heavy armor but roll for Meteorite!} \\
    \textbf{Rob}: \enquote{08 and 66. That's two hits? Nice! The meteors fall dealing 48 and 46 damage, ignoring his MARM\@. That was a nice crit! I did what, 130 damage?} \\
    \textbf{GM}: \enquote{113. (The Golem still have 115 HP) So the round ends. Roll your initiative for the second round!} \\ 
\end{multimog}

\review{Bruno: This example conflicts with the new combat rules. You may want to adjust or even redo it completely.}

\begin{multiboco}
    Optional Rule: \textbf{Quick Combat Variant}\pc%

    The combat engine presented in this chapter tries to mimic the video games' combat systems. It does, however, take time and effort to play this way, and this may not suit all groups or situations. Maybe the game is not focused on combat, or maybe there will be so much combat that using the full rules will grind things to a halt. Maybe the game will be played on a slower medium, like Play-by-Post, where even the simplest tactical combats might take weeks to happen.

    For these cases, you might want to use these simplified rules. You'll be ignoring most of the rest of the book except for flavor: there will be no mechanical need for job abilities, spells, equipment, or status effects. 

    First, add the following skills to the skill list:

    \tskill{Fencing (Earth)}: Melee fighting ability, offensive and defensive. May be used when you attack unarmed or with a melee weapon, or defend yourself from enemies attacking up close.

    \tskill{Marksmanship (Air)}: Ranged fighting ability, offensive and defensive. May be used when you attack with a ranged weapon, or defend yourself from enemies attacking with missiles.

    \tskill{Spellcraft (Fire)}: Magical prowess. May be used when you want to cast spells, use your magical power to harm or heal, or otherwise enforce your will on the world.

    \tskill{Spell Resistance (Water)}: Magical defense. May be used to counterspell, to resist magic once it has been cast, and to dispel persistent magic in the world. 

    Second, instead of gaining one skill point per three character levels, you earn one skill point for each character level. The sum of a crystal's skills must still not exceed its level.

    Third, each time you re-roll a challenge by using a skill, you ``spend'' that skill level. This represents physical, mental and emotional fatigue. Your ``spent'' skill levels return after an undisturbed night's sleep. You can always make one attempt at any skill.

    Lastly, there are no specific Combat rules. To resolve combats, pose challenges appropriate to the fight at hand using the new skills or the others, as appropriate. Like the rest of the challenge system, Destiny Points and Quirks may be used during combat.
\end{multiboco}

\begin{multiboco}
    Optional Rule: \textbf{Scaling down the Numbers}\label{optrule:scaling}\pc%

    The FFRPG 4th edition kept the d100 mechanic from earlier iterations of the Returner’s games. However, the large die can increase the burden of an already-crunchy game. This optional rule revamps the whole game engine with lower numbers, and generally speeds up play by easing the math burden, by making every roll in the game a d10 roll.

    To use it, remove all references to crystal values, keeping only crystal levels. Anything in the book that references crystal values uses cry instead. At creation you have 20 XP\@. Getting to a crystal level requries the square of that level; at leavel 4 you will have spent 16 XP and need to reach 25 XP for level 5.

    Characters' HP and MP are calculated by adding the Job bonuses to either 10 times your Earth Level (HP) or 10 times your Water Level (MP).

    Challenges use difficulties ranging from 1 to 9, with destiny points adding 2 or 4 to the roll result. Simplier dice make it easier to roll all at once and pick the best.

    For combat, divide all difficulties by ten, rounding down. For example, \taction{Attack} is crystal vs. crystal, difficulty 4. For the roll bonus, roll an extra d10, or just use 5 every time for speed.

    When you roll a natural 10 on an action able to critical hit, roll again. If the second roll is a hit, the attack is a crit.

    Abilities like Time Mage’s Wild Magic also just roll a d10. Dervish's Deadly Accuracy is essentially tripling your critical hit rate, so they can threaten a critical on an 8, 9, or 10, and then resolve the confirmation roll as other classes would. The Rogue’s Dice deals damage equal to 10 times the d10 results. The Berserker’s Unwavering Fury Ability can increase or decrease the confirmation roll’s value by 1.

    This option swaps granularity for speed, reducing the mental load on the GM and the players. It is recommended to use it if you don’t mind losing the granularity of the d100 and/or the nostalgic experience of using d100, especially if you do not have access or do not want to use computer aid during the game.
\end{multiboco}