\section{Challenges}\label{sec:challenges}
\begin{multicols}{2}
During a FFRPG 4th Edition adventure, the characters will face challenges created by the GM to succeed in their goals. A Challenge is a situation within the game, which has, necessarily, the following three characteristics:

\begin{enumerate}[label=\alph*]
\item \textbf{Failure and success chance.}

A Challenge must have a chance to fail and a chance to succeed. Walking on the flat and solid ground is not a Challenge, as it does not have chance of failure; walking in the air is also not a Challenge, because it has no chance of success. Walking on a rope in a cliff may be a challenge because it has chance of failure and success.

\item \textbf{Punishment for failure}

A Challenge must have a punishment for the failure, although it may be retried. This punishment for the failure may simply deny the characters the success’ reward. An attempt to open a locked door, using picks and thief’s tools is not a challenge if the character can keep trying again any number of times to get the door open, but may be a challenge if the room is filling with acid or if the lock breaks after the first attempt.

\item \textbf{Storytelling impact}

A Challenge must have a significant impact on the story. Riding a chocobo has chance of failure and success and has a punishment for failing (the fall). But in a 10-day trip riding a chocobo to journey between two cities, the way the character will ride does not have a significant enough impact on the story to become a Challenge. However, if the character runs away from a desert castle on fire being chased by bad guys in giant robots, the way he rides can indeed be a Challenge.
\end{enumerate}

\begin{center}
    \adjincludegraphics[width=0.45\textwidth]{block-city-future}
\end{center}

To create a Challenge, the Game Master should indicate what is the Challenge’s Skill and what is its difficulty. The Challenge’s difficulty is its failure chance. Thus, the GM decides what is the Challenge’s failure chance, indicated by a number between 1 (one) and 99 (ninety-nine). For a character succeed in this Challenge, his player must roll 1d100 (a 100-sided dice or two 10-sided die, one representing the tens and the other the singles) and the die roll should be greater than the established difficulty. For example, if the GM establishes as 30 the difficulty of a challenge, the player must understand that he has 30\% chance of failing, and consequently 70\% chance of success. When you fail in a Challenge, if you have the appropriate Skill, you may re-roll it. For each level in that Skill, you may re-roll the d100 once. Use the best result as your challenge roll. A player may always forfeit a Challenge. If he does, his character automatically fails. Notice that this does NOT means the character wants to fail. A character may still try very hard to achieve a forfeited Challenge but will fail nonetheless. When a player chooses to forfeit a Challenge, he may describe how his character fails.

\begin{boco}
Optional Rule: \textbf{General Acumen}

The game assumes that Stats have no direct relation with the character's capabilities, so a character with high Earth and no Strength Skill is not strong. Some groups prefer Stats with a greater impact on challenge. If you use this optional rule, add the Stat Level to the result of all Challenge rolls linked to that Stat's Skills.
\end{boco}
\end{multicols}
\clearpage
\section{Destiny}\label{sec:destiny}
\begin{multicols}{2}
Destiny is what distinguishes true heroes and villains from ordinary people. They are a representation of heroism and a tool for sharing the game's narrative. The Destiny Points are a Group shared currency; so, when a player earns Destiny Points, they add it to the Group’s Destiny Points; when he spends Destiny Points, these points are deducted from the Group's total.

\subsection{Earning Destiny Points}\label{subsec:earndp}
\begin{wrapfigure}{r}{0.3\textwidth}
    \adjincludegraphics[width=0.25\textwidth]{block-whitemage}
\end{wrapfigure}

There are many ways a character may earn Destiny Points. The main one is through the character's Quirks. In all Quirks, there are ways in which they create problems and force Challenges, hindering the character’s ability to achieve its goals.

The GM should provide one Destiny Point for the Group when problems arising from the Quirk create a Challenge that the characters have to overcome in order to avoid serious consequences; two Destiny Points if the problems causing is three or more concurrent or sequential Challenges that have to be overcome to prevent serious problems; or three Destiny Points if the characters cannot avoid the disastrous consequences of the problem caused by Quirk.

In addition, the Game Master may grant Destiny Points whenever the Group achieves a significant goal or according to the story. Also, the Game Master may give a Destiny Point to force a player to re-roll a Challenge, even against his will.

Lastly, whenever a player chooses to forfeit a Challenge related to one of his Quirks and the forfeited Challenge creates significant problems for the group, give him a Destiny Point. Do not give a Destiny Point if the forfeited Challenge has none or little significative story impact.

\subsection{Spending Destiny Points}\label{subsec:spenddp}
Destiny Points can be spent in various ways. The list below summarizes these forms.

\subsubsection{With Quirks}
There are two ways of spending Destiny Points with Quirks. By spending one point, the player can add 20 to a Challenge's d100 result if it is related to the Quirk, or the player may add 40 by spending two points.

The player may also wish to spend three points. If he does, the player can declare automatic success on any Challenge related with its Quirk.

Thus, a character with the Brute Quirk can use Destiny Points to add 20 or 40 to the d100 result, or even to automatically succeed in most Strength Challenges, but is unlikely to do the same in a Running Challenge. These points must be spent after rolling the die.

This rule only affects Challenges and may not be used in attack or reaction rolls during combat.

\subsubsection{Feat of Heroism}
A character may spend four Destiny points to perform feats of heroism that exceed the normal limits of human capacity. The key word here is heroism: Holding a house that would collapse on helpless children with the power of your muscles, disarming a bomb that would explode the city in the last minute, jumping from a height of twenty meters to hold on to a rope and climb to the villain’s helicopter, and so on.

A feat of heroism is always successful, even if the character does not have any applicable Quirk. After a feat of heroism, the master can’t continue the narrative with a ``no'', but at most with a ``yes, but \ldots''. ``Yes, you hold the house and prevent it from falling on the orphans, but you can’t hold it more than five minutes and someone will have to get them out of there!'' is a good answer to a feat of heroism.

\subsubsection{Getting Clues}
For one Destiny Point, the players can get a clue or a hint from the GM on a problem. The easiest way to do this is to have a nondescript citizen say a one-liner like ``Don Tonberry hate smoke!'' or ``You can’t cross the Lethe river swimming'' and then immediately disappear to where it came from. Characters with relevant Traits may also spend a Destiny Point to acquire relevant clues about their field of expertise.

\subsubsection{Divine Intervention}
By spending 7 Destiny Points, a character can dictate the outcome of an event or include things in the world. This divine intervention can only be invoked to save character(s) from an extremely dangerous situation or to increase the scene’s drama, but never to generate an anticlimax. Dictating that ``The Shadow Lord is overcome with grief and decides to kill himself'' is not an acceptable intervention but saying that ``When the group is surrounded, and everyone is about to fall into the abyss, a group of giant eagles comes and grabs the characters, flying them away!'' is a possible intervention. % chktex 36

If you use divine intervention when your character is dying, in a way that the character does not avoid his death, costs 4 Destiny Points instead of 7. Remember that using divine intervention by 4 points means the character will surely die at the end of intervention, and nothing that the characters can do will avoid this (except Cheat Death, below).

\subsubsection{Cheat Death}
For 10 Destiny Points, a player can restore a dead character to life or otherwise escape from something that surely would kill him. This ``resurrection'' is never immediate; the character will always come back only after everyone has already given up. Remember also that Cheat Death does not mean that the character will not have consequences; he may have broken bones, health problems and mental disorders that can leave him at least temporarily incapacitated. The player must decide how, exactly, the character survived (or even came back from hell itself).

\subsubsection{With Traits}
Each trait has a particular way to benefit from Destiny Point expenditure. They may be activated by spending one Destiny Point. Check the Trait list, on page~\pageref{subsec:traits}, for its description. For example, a character may spend a Destiny Point to activate his Prot\'eg\'e Trait and re-roll a Challenge related to saving his Prot\'eg\'e. 

\subsubsection{Avoiding Disadvantages}
Traits mention ways in which the character can receive penalties. If this happens, he must spend Destiny Points to avoid them. One point expenditure avoid the creation of a Challenge that the characters have to overcome in order to avoid serious consequences; two Destiny Points avoid problems causing three or more concurrent or sequential Challenges that the characters have to overcome to prevent serious problems; three Destiny Points must be spent to avoid it if the consequences of the problem caused by Trait does not need a Challenge to happen.

\begin{boco}
Optional Rule: \textbf{Solving Controversy}\pc%

When two players cannot agree on something, just have the two spend any number of Destiny Points. Whoever spends more points is right, and the other player is forced to accept his argument.
\end{boco}
\end{multicols}

\begin{center}
    \adjincludegraphics[width=0.7\textwidth]{blockwhite-cloud}
\end{center}

\section{Combat}\label{sec:combat}
\begin{multicols}{2}
Combat is the raison d’etre of many rules through this book. All rules regarding Jobs, Spells and Equipment are only tools to be used during tactical combat. The rules described below turn combative moments in a simulation that uses all the concepts discussed so far to present tactical challenges to the group. Each fight consists of rounds that follow until one side has fled, surrendered or been defeated.

\subsection{Initiative}\label{subsec:init}
At the beginning of each round, each character involved in the fight will roll 3d10 and record the values. The die roll total is his initiative and the result of each dice means the character's actions. After the initiative roll, the round will continue for 10 phases, starting from phase 1 and ending in phase 10. The phases happen sequentially, in ascending order. In phase 1, all characters who had at least one result ``1'' in one of the initiative dice may take one action for each die that shows ``1''.

If more than one character acts the same phase, the one with the highest total initiative acts first. If the characters have the same total initiative, the one with the greater Air Stat will act first. After acting, the character discards the current phase initiative die, reducing his initiative total. When there are no more characters able to act in phase 1, the round will move to phase 2 and so on. After phase 10 ends, the round ends and a new round begins.

\begin{boco}
    Optional Rule: \textbf{Team Initiative}
    
    Should you want a simpler way to track initiative, run each phase by having the all player characters do their actions first, then all enemies. Both the player characters and the enemies take actions at any desired order inside their team's turn. After all enemies have taken their action, move to the next phase. 
    \end{boco}
    

\subsection{Action Types}\label{subsec:actions}
\subsubsection{Standard Action}
A standard action takes place along the lines described above. The player uses an initiative die with value equal to the current phase to act.

\subsubsection{Interrupt Action}
This action occurs when the character can't or does not want to spend a initiative dice representing the current phase. Any character may perform any action or reaction at any phase by spending \textbf{two} initiative die of any value. Some abilities allow the character to perform actions ``as an interrupt'' by spending one initiative dice of any value in specific situations.

\subsubsection{Delayed Action}
In this case, the character chooses not to act at this phase even having an initiative die with the correct value. He may then postpone the action to a later phase. The delayed dice’s result is counted for the initiative total at its original value. A character may not delay more than one action at the same time.

\subsubsection{Free Action}
The free action occurs without the character spending an initiative dice. It occurs at specific times determined by the rule that creates it. An example of free action is talking.

\subsection{Speeds}\label{subsec:speeds}
\subsubsection{Quick Action}
A Quick action occurs when the character acts. The character discards the initiative dice and its effects are immediate. A action with charge time of zero is a quick action.

\subsubsection{Slow Action}
A Slow (X) action implies that the character must charge before performing. He declares the action as usual, and then spends (X) phases charging their action. Only after this time has elapsed, the action’s effects will happen. During this charging time, he may not react or do any other actions, but may delay their actions. If he needs to delay more than one action, as he may not delay more than one action, the extra actions are lost. All actions with a charge time of 1 or greater are Slow actions. Some effects may increase or decrease the charge time of your actions, changing the number of phases you need to charge the Slow action.

In addition, if the Slow action require you to prepare your action beyond phase 10, you lose all non-delayed initiative die this round and, in the next round, roll one fewer initiative dice. At the phase when you finish the preparation, the action’s effects happen as usual. For example, a character initiates a Slow (7) action in phase 6. In the next round, he rolls one fewer initiative dice and in Phase 3, the action’s effects happen.

\subsubsection{Reaction}
Reactions occur when the character uses specific abilities. They interrupt other actions and must be resolved before the first action’s effects are applied. To use a reaction, the character can spend an initiative dice with the current phase’s value, use a delayed action or even perform an interruption. Some reactions are free actions. A character may wait to see if an action is successful or not before declaring his reaction.

\subsection{Attacks and Rolls in Combat}\label{subsec:attacks}
All rolls in combat (attack rolls, reactions, Spells, etc.) have the following characteristics: One Offensive Stat, one Defensive Stat and a difficulty. For example, the \taction{Attack} action using a Bow is an Air (Offensive stat) vs Air (Defensive stat) attack, difficulty 40. The roll is as follows: The player will roll 1d100 and add his Offensive Stat’s value. He will be successful if the result is higher than difficulty + the target’s Defensive Stat’s value. If target choose to not resist the action, it is always successful, unless the action says otherwise.

If the attack is against a group (be it ally’s or enemy’s), the player must perform only one roll. After adding the d100 roll to Offensive Stat’s value, it should compare it separately with the sum of difficulty + Defensive Stat’s value of each target and may be successful against only part of the opponents.

If the attack deals damage or recover HP or MP, the d100’s unit digit will be added to the damage dealt or value healed, unless the attack says otherwise. Finally, if a character re-rolls an attack, either because he can re-roll one of his attacks, or because the target can force his opponent to re-roll his attacks, the character can look at the results and choose the best (or be forced to choose the worst).

Each attack, except Spells, may be Ranged or Melee. All Ranged attacks are noted as such; every non-Ranged attack is Melee. Flying enemies may not be hit by Melee attacks, unless the attacker is also Flying. Spells and Reactions aren't neither Ranged nor Melee and may target Flying enemies normally.

\subsection{Basic Actions and Reactions}\label{subsec:basicactions}
All characters can perform the following actions, regardless of equipment, Job or Abilities.

\subsubsection{\taction{Attack}}
Quick action. If the character is unarmed, perform an Earth vs Earth attack, difficulty 70. The damage is physical, \telem{Crush}-elemental, equal to Earth level. If it is equipped with a weapon, use the weapon’s Offensive and Defensive Stats, difficulty 40, and deal weapon damage. The Wealth and Items section, below, has more details about the weapons between pages 94 to 107. This action may score critical hits, dealing twice damage.

\subsubsection{\taction{Dodge}}
Reaction. Use when suffering a physical attack. Roll Air vs Earth, difficulty 70. If successful, you don’t suffer the attack’s effects.

\subsubsection{\taction{Draw}}
Quick action. No roll needed. You draw and ready a weapon or inventory item to use it.

\subsubsection{\taction{Item}}
Quick action. No roll needed. You use a drawn item or replace the equipped weapon with a drawn weapon. In this case, the exchanged weapon is stored in your inventory as a free action.

\subsubsection{\taction{Flee}}
Quick action. Roll Air vs Air, difficulty 40. If successful, you run away from combat. Use the opponent with the highest Air Stat as the target of this action.

\subsection{Critical Hits}
Various actions can score critical hits. Unless the equipped weapon or the Ability says otherwise, critical hits double the damage. To score a critical hit, the roll result should be two identical numbers (100, 99, 88, 77, 66, 55, etc.) and the attack must be successful. If an action does not state that it might achieve critical hits, rolling identical numbers does no extra effect.

\begin{center}
    \adjincludegraphics[width=0.3\textwidth]{block-suplex-train}
\end{center}

\subsection{Elements and Damage}
All damage has one element. However, by itself, the element does not influence the damage, unless the character who receives the damage absorbs, or is resistant, immune or vulnerable to elemental damage. The element list is: CRUSH, PUNCTURE, CUT, FIRE, ICE, LIGHTNING, AIR, EARTH, WATER, BIO, LIGHT and SHADOW.\@{}%

Regardless of the damage dealt, the attack roll’s d100’s singles digit adds to damage, assuming 0 as 10. If, for example, a character uses an attack that deals 20 damage and rolls 63 on the attack, hitting its target, it will deal 23 damage.

Unless the attack says otherwise, all damage is reduced by the target's ARM, if the attack is physical, or the target’s MARM, if the attack is magical. The damage suffered after reducing by Armor or Magic Armor, if any, is deducted from the target’s current HP.\@{}There is no penalty for having current HP lower than max HP, unless current HP is 0. In this case, the character falls unconscious and may not perform actions while his HP is lower than 1. Lastly, remember to always round down.

Various effects may change the damage dealt. When in doubt, follow the sequence below:
\begin{enumerate}
\item \textbf{Calculate Base Damage}

Base damage is calculated by multiplying the damage factor by the relevant Stat Level.

\item \textbf{Apply Action Modifiers}

Some actions do 150\%, 200\%, 75\% or any other modifier to the Base damage. Multiply this modifier to the Base Damage.

\item \textbf{Account for Strengthen and Weaken}

Strengthen (Physical or Magic) may increase the damage by 25\%. Weaken (Physical or Magic) may decrease the damage by 25\%.

\item \textbf{Add the roll's singles digit}

Add the roll's singles digit to the damage, assuming 0 as 10.

\item \textbf{Reduce by target's ARM or MARM}

Reduce the damage dealt by the enemy's ARM, if physical, or MARM, if magical.

\item \textbf{Apply target's Modifiers}

Critical Hits and modifiers like the Shell and Protect status, any elemental weakness or resistance, or even some Action modifiers must be applied after accounting the enemy's defense.
\end{enumerate}

\begin{boco}
Dealing damage is the most time-consuming part of combat. If a group does not address this problem, it may turn combat into a tedious math exercise. Some tips to speed it:

First, do steps 1 to 3 before you even land a blow. Having your damage pre-calculated works wonders to speed up combat. Instead of writing your damage as \taction{Jump}, (200\%) 5x Earth, note it as \taction{Jump}, 80 damage. Also, account for Strengthen and Weaken as soon as you receive the status, not when you decide to attack. Try to do your math while the other players are describing their actions.

Second, remember that 50\% is one half, and 25\% is half of a half. 10\% is the number, ignoring the unit digit. Round down your calculations to speed them. To find a quarter of 138, for example, halve it first (69) then halve it again (34). 150\% is one plus half; 125\% is one plus a quarter; and 75\% is one minus a quarter.

Lastly, try to cancel opposite modifiers, even if the numbers aren't exactly equal. A \taction{Mighty Blow} or Critical Hit against an enemy with Protect deals normal weapon damage; If you use \taction{Guardbreak} on a Vulnerable enemy, it also deals normal weapon damage; and so on. Keep in mind that speed trumps math accuracy.
\end{boco}

\subsection{Healing}\label{subsec:heal}
After an undisturbed night's sleep, restore your current HP and MP values to max value. This happens even to unconscious characters. In other moments, healing provided by any effects doesn’t affect unconscious characters, unless its description specifically says that the effect affects targets with zero HP.\@{}Similarly, an effect that specifically targets zero HP characters doesn’t have any effects on a character with current HP equal to 1 or more.
\end{multicols}

\begin{multiboco}
Optional Rule: \textbf{Quick Combat Variant}

Tactical Combat, as presented in this chapter, is a deep strategical endeavor that tries to mimic the video game's combat systems. However, there are lots of examples where the group does not want to spend so much time and effort to use these rules. Maybe the game is not focused on combat, or maybe there will be so much combat that using the full rules will grind things to a halt. Maybe the game will be played on a slower medium, like Play-by-Post, where even the simplest tactical combats might take weeks to happen.

In this case, you might want to use this variant rule. If you wish to use it, ignore most of the Character Options chapter: you'll have no need for Jobs, Abilities, Spells, Equipment or Status Effects. Your characters might have a Job or two, but only for flavor.

First, add the following Skills to the Skill list:

\tskill{Fencing (Earth)}: Melee fighting ability. Including both attacking with melee weapons and defending against them. May be used in situations where you must attack with a melee weapon or defend yourself from enemies attacking in close range.

\tskill{Marksmanship (Air)}: Ranged fighting ability. Including both attacking with ranged weapons and defending against them. May be used in situations where you must attack with a ranged weapon or defend yourself from enemies attacking with missiles.

\tskill{Spellcraft (Fire)}: Magical prowess. Ability to use magic to enforce your will. May be used in situations where you want to cast spells and use your magical power to harm or heal.

\tskill{Spell Resistance (Water)}: Magical defense. Ability to overcome enemy magic. May be used in situations where you want to dispel enemy magic or simply resist its effects. Also used to counterspell.

Second, instead of gaining one Skill point per three character levels, you earn one Skill point for each character level. You still may not spend more Skill points in all Skills related to a Stat than levels you have in that Stat.

Third, each time you re-roll a Challenge due to using a Skill, you ``spend'' that Skill level. This represent physical, mental and emotional fatigue. Your ``spent'' Skill levels return once you have the time to rest 8 hours.

Lastly, there are no specific Combat rules. To resolve combats, use Challenges appropriate to the action at hand, using these new Skills or the core Skills, as situation demands. Unlike the tactical combat rules, Destiny Points and Quirks may be used during combat in this variant, as it is handled by normal Challenge rules.
\end{multiboco}

\begin{multimog}
Later in his adventures, JBMog, now a \ordinalnum{30} level character, was in a dungeon with his friends Rob, a \ordinalnum{28} level Warrior/Rune Knight and Nyarly, a \ordinalnum{31} level Adept/Wizard. After an unfortunate failed Challenge roll, they fell into a trap, activating an iron Golem guardian. \\
\textbf{GM}: \enquote{Start of Round 1. The creature rolls 1, 3 and 6 as initiative. Roll your initiative!} \\
\textbf{JB}: \enquote{5, 6 and 9. I use my Preemptive Strike to change the 9 to 1. My initiatives are 1, 5 and 6!} \\
\textbf{Rob}: \enquote{3, 3, and 8. And Nyarly rolled 4, 5, and 10} \\
\textbf{GM}: \enquote{Phase 1. You're first JB, then Golem acts.} \\
\textbf{JB}: \enquote{I'll strike the golem with my lance. I rolled a 61, plus my Air is 148 total. It hits?} \\
\textbf{GM}: \enquote{You try to attack him with your polearm, but the heavy armor of the golem deflects it (The attack targets Earth plus diff 40. His value is 112, so he needed to overcome 152 to hit). He ignores your attacks while he prepares his own. He starts charging a Slow axe attack against Nyarly. We begin Phase 2, and his attack is finished. I roll a 34 (plus his 112 Earth totals 145) for a total 114 damage (110 damage plus 34's singles digit).} \\
\textbf{Nyarly}: \enquote{Yikes! I will use an interrupt action and react to use \taction{Will Shield}. I spend the 4 and 10 dice. I rolled a 69, for a total of 149. I made it? (The GM nods, as \taction{Will Shield} have a 10 difficulty and the Golem's Earth value is only 112.) Nyarly creates a magical shield that blocks the blow and spends 14 MP (10\% of his 143 MP).} \\
\textbf{Rob}: \enquote{Seems no one got actions in Phase 2, so I'm acting at phase 3. Seems that his weak spot is Air, so my first action is to draw my Meteo Knuckle. My second action (Rob can act before the golem since his initiative total is 11 and the Golem's is 9) is to use \taction{Item} to stash my Old Axe and equip the gloves.} \\
\textbf{GM}: \enquote{The Golem starts to leak out a strange gas. It attacks all characters, rusting your equipment! Who got Water lower than 43 is hit with the Weaken (Armor) status until the end of the next round.} (The GM rolled a 31. 31 plus the Golem's Fire value is 113, and the attack targeted the group and had diff 70) \\
\textbf{JB}: \enquote{I think the only one hit was Rob (Rob nods). My turn now, right? (Nyarly's initiative total is 5, while JB's is 11) I'll use \taction{Advice} on Rob. I want him to crit with that knuckle!} \\
\textbf{Nyarly}: \enquote{So, now it is my turn. I'll spend 60 HP to unleash a \taction{Fury Brand}. Rolled 90 for a total of 181 vs Water+40. If it hits, that's 131 Fire damage and Berserk.} \\
\textbf{GM}: \enquote{Yeah, that hits. Your staff burns with your magical fire when you sacrifice your lifeforce to summon a cleansing flame. Your strike hits true, dealing 101 damage (that's 131 minus the 30 MARM), but the creature seems to be immune to your mental effect. In the golem's turn at phase 6, he delays his action.} \\
\textbf{JB}: \enquote{I'll charge Geomancy. No actions until phase 8? Ok. Let me roll for Geomancy! Hm\ldots 75? What’s the Major effect for Underground? Ah, Cave In. I'll spend the 35 MP to increase the damage. (JB rolls a 51 and hits) That's 101 damage (100 damage plus 51's singles digit)! Take that!} \\
\textbf{GM}: \enquote{Rocks fall, and the Golem receives 71 damage from the cave in, but he's still rocking!} \\
\textbf{Rob}: \enquote{Phase 8? At last! I'll attack him\ldots Oh, only 11\ldots I miss\ldots No, wait, I crit due to JB!\@{}Thanks! I'll do-} \\
\textbf{GM}: \enquote{Actually the Golem reacts with his delayed action. He rolls a 17 to block and/ldots (checks Rob's Earth value of 111) fails. Your critical does only 19 damage due to the golem's heavy armor but roll for Meteorite!} \\
\textbf{Rob}: \enquote{08 and 66. That's two hits? Nice! The meteors fall dealing 48 and 46 damage, ignoring his MARM.\@{}That was a nice crit! I did what, 130 damage?} \\
\textbf{GM}: \enquote{113. (The Golem still have 115 HP) So the round ends. Roll your initiative for the second round!} \\ 
\end{multimog}

\begin{multiboco}
Optional Rule: \textbf{Scaling down the Numbers}\label{optrule:scaling}

The FFRPG 4th edition kept the d100 mechanic from earlier iterations of the Returner’s games. However, the dice used increases the burden of an already-crunchy game. This optional rule revamps the whole game engine with lower numbers, and generally speeds up play, easing the math burden.

To use it, remove all references to Stat Values, keeping only Stat Levels. Anything in the book that references Stat Values uses Stat Levels instead. At creation you have 20 XP.\@{}A Stat level of 1 costs 1, 2 costs 4, 3 costs 9, 4 costs 16, 5 costs 25, and so on. You just keep any XP you don't spend (so a 3 2 2 1 character starts with 2 XP leftover).

Character’s HP and MP are calculated by adding the Job bonuses to either 10 times your Earth Level (HP) or 10 times your Water Level (MP).

Challenges use a d10 instead of a d100, with difficulties ranging from 1 to 9. Skills works as usual, and Destiny Points adds 2 or 4 to a dice, instead of 20 or 40. When rolling Challenges, you can just roll all die at once and pick the best result.

Combat also uses a d10 instead of d100. Divide all difficulties by ten, rounding down if necessary. \taction{Attack} Actions, for example are Stat vs. Stat, difficulty 4. When calculating damage, instead of adding the 1's digit die, you just add 5 damage.

When you roll a natural 10, on an attack able to critical hit, roll again. If the second roll is a hit, the attack is a crit.

Abilities like Time Mage’s Wild Magic also just roll a d10. Dervish's Deadly Accuracy is essentially tripling your critical hit rate, so they can threaten a critical on an 8, 9, or 10, and then resolve the confirmation roll as other classes would. The Rogue’s Dice deals damage equal to 10 times the d10 results. The Berserker’s Unwavering Fury Ability can increase or decrease by 1 the confirmation roll’s value.

This option swaps granularity for speed, reducing the mental load on the GM and the players. It is recommended to use it if you don’t mind losing the granularity of the d100 and/or the nostalgic experience of using d100, especially if you do not have access or do not want to use computer aid during the game.
\end{multiboco}