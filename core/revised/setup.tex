\section{Group Creation}\label{sec:group}
\begin{multicols}{2}
An FFRPG 4e Adventure or Campaign is centered around the Group: the organization or party which the player characters are a part of, whose ideals or structure brings the players much of their general goals or objectives. Group creation is a joint activity between the players and GM and will serve to guide subsequent creation of individual player characters. This activity is very important because it defines what kind of stories will be played. To create the group, follow the 5 steps below.
\begin{enumerate}

\item \textbf{Choose Traits}

These special features are the mechanical elements that will influence the way the game is played. Each group has three Traits, chosen from the list below. Traits strongly influence how the game will unfold and should be chosen very well by the GM and the players.

\item \textbf{Choose a Name}

Self-explanatory. The Group may have a name like Returners, Knights of Ivalice, Heroes of Light, Shin-Ra Inc., SOLDIER, or any other name.

\item \textbf{Create the Roots}

What links the group? Why was it founded? What is its reason for existence? Are they the underground resistance to an oppressive empire? Are they members of a kingdom or a corporation? Are they young people from a village in the countryside?

\item \textbf{Create the Evil}

Who is the main antagonist? Who threatens the Group’s existence? Remember that this antagonist may (and probably will) change during the course of a campaign.

\item \textbf{Write the starting Destiny Points}

Destiny Points is a Group feature that can be used by players to influence the story. Starting Groups begin with 4 Destiny Points.
\end{enumerate}
\subsection{Traits}\label{subsec:traits}
\subsubsection{Mercenary}
This group was founded to acquire wealth and power. You gain experience points (XP) by earning Gil or other material possessions.

\textbf{The Good}: You have contacts that appear in the most improbable moments. By spending Destiny Points in any situation, you can find someone interested in buying or selling you things, except if this imply risk of life. However, they may not always charge fair prices, even if this merchant is a cat-man in the middle of an inhospitable mountain.

\textbf{The Bad}: People tend not to rely on mercenaries and will doubt your intentions when they know your motivations, unless you spend Destiny Points.

\subsubsection{Monster Hunter}
This group was founded to kill monsters. You gain experience points (XP) by winning battles against beasts and other creatures.

\textbf{The Good}: You may spend Destiny Points to discover things about monsters.

\textbf{The Bad}: Monsters are always hostile to you. You need to spend Destiny Points to prevent a monster from being automatically hostile.

\subsubsection{Nemesis}
This group was founded to fight something or someone of great power. This enemy can be a person or organization. This Trait can be selected more than once, each time representing a different enemy. You gain experience (XP) preventing the plans of your Nemesis or defeating him or his lackeys.

\textbf{The Good}: As much as your Nemesis hate you, it always seems to leave a hole in his plans. Whenever you are in a situation of imminent defeat to your Nemesis, you may spend Destiny Points to figure out a way to escape, in order to face it again later. This does not count as defeating or preventing the nemesis’s plan in any way: you only save yourself.

\textbf{The Bad}: Your Nemesis know your plans better than anyone. Whenever you try to spend Destiny Points to get any advantage over your Nemesis, you will need to spend twice as many Destiny Points.

\subsubsection{People’s Hero}
This group was founded to liberate people from tyranny. You gain experience points (XP) by removing corrupt officials, protecting the public and doing good deeds.

\textbf{The Good}: You may spend Destiny Points to call for assistance of the population. This help may involve shelter, food, hiding, and other support within the reach of the common man.

\textbf{The Bad}: You can’t refuse a request for help from a humble man, unless you spend Destiny Points.

\subsubsection{Protgégé}
This group was founded to protect something or someone from harm. Although it is very important for several reasons, this protégé is unable to defend itself from harm, which can be physical or not. This Trait can be selected more than once, each time representing a different entity to be protected. You gain experience points (XP) by avoiding harm for the entity or by restoring it.

\textbf{The Good}: If your protégé is in danger, you may spend Destiny Points to gain a second chance on any Challenge that can save it.

\textbf{The Bad}: If the protégé is destroyed or killed, even if is possible to reconstruct, resurrect or any other way restore it, you lose Destiny Points.

{\centering %
    \adjincludegraphics[width=0.45\textwidth,center]{block-alchemist}%
}

\subsubsection{Relics from the Past}
This group was founded to investigate the secrets of the past, either arcane or technological (depending on the campaign). You gain experience points (XP) by investigating ruins, discovering elder tomes or other ancient artifacts.

\textbf{The Good}: You may spend Destiny Points to find out stories about artifacts and other ancient legends.

\textbf{The Bad}: Many of the ancient things have profound and dire stories, carrying curses. You may suffer the curse that was upon something found, unless you spend Destiny Points. \pc%

\subsubsection{Reputation}
This group was founded to earn fame and success. You gain experience points (XP) when you can spread your reputation and become better known and loved. Alternatively, you may decide that your desired reputation is bad reputation and your goal is to become feared and hated.

\textbf{The Good}: Your reputation precedes you. You may spend Destiny Points to impress or even influence people based on your reputation.

\textbf{The Bad}: It's hard to go incognito. When you really need to be undercover, you need to spend Destiny Points, or else you will be recognized or otherwise affected by your reputation.

\subsubsection{Sense of Duty}
This group was founded to follow an organization. Define what is the specific organization, which may be a church, a kingdom, the army, a corporation, or something else. This Trait can be selected more than once, each time representing an affiliation to a different organization. You gain experience points (XP) by performing missions for the organization.

\textbf{The Good}: You may spend Destiny Points to receive help from the organization. This special aid will depend on the chosen organization.

\textbf{The Bad}: You can’t refuse missions from that organization, even if it goes against your character’s beliefs, unless you spend Destiny Points.
\end{multicols}
\clearpage
\begin{multimog}
Group creation is really a huge part of campaign creation. When planning the campaign, you may create the Group (and its Traits) ahead of time and simply present it to the players. However, if you can, just present the setting and try to create the Group along with your players. Below I'll present two different Group examples, all within the same setting. \pc%

\textbf{Setting}: Wars of Mana. Taken straight from Seiken Densetsu III (or Trials of Mana) storyline, Wars of Mana takes place on a high fantasy medieval world, where six great nations fight over control for the Mana, the magical energy that permeates the world. \pc%

\textbf{\ordinalnum{1} Group's Traits}: Relics from the Past, Nemesis (Ganelon) \& Monster Hunter. This group was created to find a way to stop Ganelon, an evil shapeshifter who is using political connections, doppelganger minions and his cunning to force the world into widespread chaos and war. They believe the key to stop them is solving the enigma of the finite Mana, discovering a way to please all the kingdoms. This campaign went with a political undertone, with the PCs struggling to uncover Ganelon's emissaries and stop his machinations. \pc%

\textbf{\ordinalnum{2} Group's Traits}: Relics from the Past, Reputation \& Nemesis (The One without a Name). The players decided they would face a timeless being from another dimension. A creature so terrible and alien that no mortal could even speak his name without delving into the pits of madness. This was even stronger due to one of the PCs being half-mad due to an encounter with it. And what’s worse, it seems that no one believed them. This campaign had a very dark horror tone, as the PCs struggled to earn a Reputation and prove to the world that an unseen threat exists. The only ones who listened were the already-corrupted cultists of this Evil God.
\end{multimog}
\vspace{\stretch{1}}
\adjincludegraphics[width=0.9\textwidth,center]{block-snow-building}
\vspace{\stretch{1}}
\clearpage
\section{Character Creation}\label{sec:creation}
\begin{multicols}{2}
Each player character has its own characteristics. They are individually assigned by the player to his character, based on the concept the player wants to roleplay. To create a player character, do the following steps:
\begin{enumerate}
\item \textbf{Choose a Name}

Again self-explanatory. Choose a name that suits the character you want to play.

\item \textbf{Choose your Traits and Quirks}

Each character has a total of 3 Traits and 3 Quirks. 2 of his Traits must be chosen from his Group’s Traits, and the last one is chosen from the Trait list, and don’t need to be one of the Group's Traits. The 3 Quirks must be chosen from the list in the~\nameref{sec:quirks} section starting at page~\pageref{subsec:quirklist}. Traits define how the character will earn experience points (XP) and evolve during the game, while the Quirks indicate how it can earn Destiny Points. Remember that the Destiny Points are shared by the Group, hence all Destiny Points income and expenses will come from the Group's total Destiny, rather than being individual characters’ values. In addition, all Traits grant ways in which the characters may spend Destiny points.

\item \textbf{Choose your Jobs}

Each player character has two Jobs, chosen from two different lists. The combination of Main and Secondary jobs can allow for a great variety of characters, each with several unique ways of contributing in battle, by wielding unique Abilities against their foes. The jobs are the following: \\
\textbf{Main Jobs}: Adept, Archer, Artist, Black Mage, Druid, Freelancer, Monk, Time Mage, Rogue, Warrior, White Mage \\
\textbf{Secondary Jobs}: Alchemist, Berserker, Defender, Dervish, Fencer, Rune Knight, Phalanx, Squire, Wizard.

More details on Jobs and their Abilities are in the~\nameref{sec:jobs-summary} section starting at page~\pageref{sec:jobs-summary}.

\item \textbf{Spend XP to increase your Stats}

In FFRPG 4e, each character has four Stats, each related to a crystal: Earth, Air, Fire and Water. More details on these Stats are in the~\nameref{subsec:stats} section at page~\pageref{subsec:stats}. A starting character has a total of 200 (two hundred) experience points (XP) to spend on their Stats. 

\item \textbf{Assign your Skills}

A character earns 1 Skill point for each 3 Levels he earns. Remember that the total Character Level is the sum of his Stat Levels. These skill points may be spent as the player wish between the skills, but the Stat Level is the maximum amount of skill points that may be spent in all skills linked to that Stat. The skills are: \\
\textbf{Earth Skills}: Strength, Climbing, Swimming, Intimidation, Tolerance, Jumping. \\
\textbf{Air Skills}: Running, Stealth, Piloting, Riding, Thievery, Acrobatics. \\
\textbf{Fire Skills}: Infiltration, Perception, Medicine, Survival, Technology, Wisdom. \\
\textbf{Water Skills}: Willpower, Bluff, Handle Animal, Charisma, Performance, Magic.

The Skill details start at page~\pageref{subsec:skills} in the~\nameref{subsec:skills} section.

\item \textbf{Acquire your Abilities}

Each Job offers Core Abilities and Specialties. A character acquires all Core Abilities they qualify for and may select one Specialty for each of their Core Abilities if they meet the required Stat levels. For more details on Abilities see the~\nameref{sec:jobs-summary} section starting at page~\pageref{sec:jobs-summary}. 

\item \textbf{Buy your Starting Equipment}

During character creation, each player can spend 200 Gil in equipment and items. It is recommended that you buy at least one weapon for your starting character. More details on wealth and equipment may be found on the~\nameref{sec:inv-wealth} section, starting at page~\pageref{sec:inv-wealth}.

\item \textbf{Finishing Touches}

Calculate your HP by adding your Earth value to your Job HP bonus and your MP by adding your Water value to your Job MP bonus using your Job's guidelines. Don't forget to flesh out the character concept using all the cues you've been collecting thus far (Traits, Quirks, Jobs, Skills, etc.). Take notes on your character backstory, motivations, personality and appearance, as that may help you roleplay your character to its fullest potential.

\end{enumerate}
\end{multicols}
\clearpage
{\centering%
    \adjincludegraphics[width=0.075\textwidth,center]{../img/common/boco}%
}
{\color{bocoblue}\small%
Optional Rule: \textbf{Skilled Rookies}

Should you want to have starting characters with more skills so the players have more can express their character concept with more skill choices early, give them 4 skills points at character creation plus 1 extra skill point for each 4 levels (instead of 1 point per 3 levels). This should give a starting character 6 skill points instead of 2, and will level out by level 48. You're trading more skills at high (49 and over) levels for more skills at lower (48 and under) levels.%
}

\begin{multimog}
JB wants to create a cute-but-dangerous Geomancer moogle. He begins with his name, of course JBMog, and looks at his Group's Traits. The Returners, as his group calls itself, have the Nemesis (Empire), People's Hero and Sense of Duty (Banon) traits. JBMog doesn't like that Banon guy, so he decides to have the Nemesis (The Empire) and People's Hero Traits. To round it up, he fetches one last trait: Protégé (Narshe Mines): He'll fight to protect his people at the Narshe mines. Looking at the Quirks, he quickly comes with a good idea of his character: The Moogle is mandatory for him, but the Feral and Fast Quirks also round up his character.

All in all, he's a quick moogle who loves animal company, but he's a tad shy in human lands. Then, he notes his jobs. Druid is a quick choice, for a Geomancer character, but he takes a minute to ponder about Secondary Jobs. After some debate, he ends up choosing Fencer to focus on the defense.

Next, the Stats. With his 200 XP, and looking at the stats, he decides to make Fire his primary stat, spending 90 XP on that one. This nets him Fire 30, as shown in the experience table. Then, he proceeds to put 10 XP in Water and 40 XP in Earth: bringing these Stats to 10 and 20, respectively. The last 60 XP he proceeds to put in Air, granting him Air 24.

His levels are Earth 2, Air 2, Fire 3 and Water 1. With these scores, his character level is 8 (2+2+3+1). For Skills, he decides to grab Perception and Performance.

As for Abilities, he notes down the first level Core Abilities: Nature's Path and Awakened (Geomancer) from the Druid, and Block Projectiles from the Fencer. Due to his Stats, he also chooses a Specialty: Nature Warrior (Polearms \& Bows), because what’s better than a Polearm-wielder Moogle? He hasn't leveled up enough to get other Specialties, so he goes to Equipment.

Being a Nature Warrior, he proceeds to get the heaviest armor around: a Leather Plate (99 Gil) and an Iron Spear (63 Gil). With the remaining Gil he buys a Tonic (25 Gil) and pockets the remaining money (13 Gil). Finally, he does his HP (level x4 = 8x4 = 32 + earth (20) totals 52 HP) and MP (level x1 = 8x1 = 8 + water (10) totals 18) calculations.

So, his character sheet is done! Let's see how it is:
\begin{center}\label{tab:jbmog}
  \rowcolors{1}{}{gray!10}
  \begin{tabular}{cccc}
    \toprule
    \multicolumn{4}{c}{JBMog, 8th level Druid/Fencer} \\
    \textbf{Stat}  & \textbf{Level} & \textbf{Value} & \textbf{XP Spent} \\ \midrule
    \textbf{Earth} & 2              & 20             & 40                \\
    \textbf{Air}   & 2              & 24             & 60                \\
    \textbf{Fire}  & 3              & 30             & 90                \\
    \textbf{Water} & 1              & 10             & 10                \\ \bottomrule
  \end{tabular}
\end{center}

\noindent \textbf{Skills}: \tskill{Perception} 1 and \tskill{Performance} 1.\\
\textbf{Traits \& Quirks}: \ttrait{Nemesis (The Empire)}, \ttrait{People's Hero}, \ttrait{Protégé} (Narshe Mines); \tquirk{Moogle}, \tquirk{Feral}, and \tquirk{Fast}.\\
\textbf{HP} 52/52; \textbf{MP} 18/18; \textbf{ARM} 3; \textbf{MARM} 0\\
\textbf{Equipment}: \tequip{Iron Spear} (Air vs Earth, 6 damage), \tequip{Leather Plate}\\
\textbf{Abilities}: \tability{Nature's Path [Nature Warrior]}; \tability{Awakened (Geomancer)}; \tability{Block Projectiles}\\
\textbf{Actions}: \taction{Geomancy}, \taction{Arrow Guard}
\end{multimog}