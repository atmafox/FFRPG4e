\label{subsec:pjob-freelancer}
\begin{jobdesc}[name=pjob-freelancer]
    The Freelancer is a dilettante, forgoing all tried and true paths to forge their own road. It embodies versatility, being able to dabble into all other Main Jobs, building unique battle strategies and tactics. With it, you’ll be able to pick and choose your skills, mixing and matching strategies and changing your Job almost at-will. \pc

    \textbf{Representatives}: Onion Kid (FF III), Bare Job (FF V) \pc

    \jobstats[hpa=3x,hpb=4x,hpc=5x,hpd=6x,mpa=1x,mpc=2x,armor=?,weapons=?]
\end{jobdesc}

\begin{ffminipage}
{\centering \textbf{Abilities}\par }

\noindent\tability{Versatility}: Core Ability acquired at level 1. Your HP bonus is 3x your level, and increases by 1x your level at levels 15, 30 and 60. Your MP bonus is 1x your level, and increases by 1x your level at level 30.
\end{ffminipage} \pc \begin{ffminipage}
\noindent\tability{Job Change}: Core Ability acquired at level 1. Choose a Main Job. You undergo a Job Change to the chosen Job. You can equip the weapons and armor as if you had the first Ability of this Job. Each time you spend Job Points, select abilities from a job and Job Change to that job. Whenever you make a Job Change, treat as you had the new Job to determine which weapons and armor you may equip, instead of the previous. You may only keep the option to equip a piece of gear if you have an Ability that permanently grants access to a weapon or armor type. Your Main Job’s abilities other than \tability{JP UP} may not change your HP and MP bonuses. Lastly, you may not acquire any Main Job’s Specialties unless by spending Job Points.
\end{ffminipage} \pc \begin{ffminipage}
\noindent\tability{JP UP}: Ability acquired at level 8. Gain 3 Job Points (JP). You may spend them accordingly to the table below. You may spend less than your total Job points but cannot spend them again until you gain more Job Points. To gain any Core Ability or Specialty this way, you must satisfy all the Ability's required levels.
\begin{center}
    \rowcolors{1}{gray!10}{}
    \begin{tabular}{lcl}
        \toprule
        \textbf{Ability Gained} & \textbf{JP Cost} & \textbf{Special} \\ \midrule
        Increase your HP bonus multiplier by 1 & 1 & May be chosen once \\
        Increase your HP bonus multiplier by 1 & 2 & May be chosen twice \\
        Increase your MP bonus multiplier by 1 & 2 & May be chosen twice \\
        \tability{Awakened}, \tability{Natural Domain}, \tability{Arcane Devotion} core abilities & 2 & \\
        Pick a job's first ability that grants spell groups & 2 & Other than \tability{Arcane Devotion} \\
        Pick a job's second or later ability that grants spell groups & 1 & Other than \tability{Arcane Devotion} \\
        Pick a specialty from an ability you already have a specialty from & 2 & \\
        Pick a core ability or specialty not listed above & 1 & \\ \bottomrule
    \end{tabular}
\end{center}
\end{ffminipage}

\noindent\tability{JP UP}: Core Ability acquired at level 16. Gain 1 Job Point. You may spend any Job Points you have stored immediately.

\noindent\tability{JP UP}: Core Ability acquired at level 23. Gain 1 Job Point. You may spend any Job Points you have stored immediately.

\noindent\tability{JP UP}: Core Ability acquired at level 30. Gain 1 Job Point. You may spend any Job Points you have stored immediately.

\noindent\tability{JP UP}: Core Ability acquired at level 37. Gain 1 Job Point. You may spend any Job Points you have stored immediately.

\noindent\tability{JP UP}: Core Ability acquired at level 44. Gain 1 Job Point. You may spend any Job Points you have stored immediately.

\noindent\tability{JP UP}: Core Ability acquired at level 51. Gain 1 Job Point. You may spend any Job Points you have stored immediately.

\noindent\tability{JP UP}: Core Ability acquired at level 58. Gain 1 Job Point. You may spend any Job Points you have stored immediately.

\noindent\tability{JP UP}: Core Ability acquired at level 65. Gain 1 Job Point. You may spend any Job Points you have stored immediately.

\adjincludegraphics[width=.75\textwidth,center]{block-airship}
