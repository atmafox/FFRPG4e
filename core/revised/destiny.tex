\section{Destiny}\label{sec:destiny}
\begin{multicols}{2}
Destiny is what distinguishes true heroes and villains from ordinary people. They are a representation of heroism and a tool for sharing the game's narrative. The Destiny Points are a Group shared currency; so, when a player earns Destiny Points, they add it to the Group’s Destiny Points; when he spends Destiny Points, these points are deducted from the Group's total.

\subsection{Earning Destiny Points}\label{subsec:earndp}
\begin{wrapfigure}{r}{0.3\textwidth}
    \adjincludegraphics[width=0.25\textwidth]{block-whitemage}
\end{wrapfigure}

There are many ways a character may earn Destiny Points. The main one is through the character's Quirks. In all Quirks, there are ways in which they create problems and force Challenges, hindering the character’s ability to achieve its goals.

The GM should provide one Destiny Point for the Group when problems arising from the Quirk create a Challenge that the characters have to overcome in order to avoid serious consequences; two Destiny Points if the problems causing is three or more concurrent or sequential Challenges that have to be overcome to prevent serious problems; or three Destiny Points if the characters cannot avoid the disastrous consequences of the problem caused by Quirk.

In addition, the Game Master may grant Destiny Points whenever the Group achieves a significant moment in the story. Also, the Game Master may give a Destiny Point to force a player to re-roll a Challenge, even against his will.

Lastly, whenever a player chooses to forfeit a Challenge related to one of his Quirks and the forfeited Challenge creates significant problems for the group, give him a Destiny Point. Do not give a Destiny Point if the forfeited Challenge has none or little significative story impact.

\subsection{Spending Destiny Points}\label{subsec:spenddp}
Destiny Points can be spent in various ways. The list below summarizes these forms.

\subsubsection{With Quirks}
There are two ways of spending Destiny Points with Quirks. By spending one point, the player can add 20 to a Challenge's d100 result if it is related to the Quirk, or the player may add 40 by spending two points.

The player may also wish to spend three points. If he does, the player can declare automatic success on any Challenge related with its Quirk.

Thus, a character with the Brute Quirk can use Destiny Points to add 20 or 40 to the d100 result, or even to automatically succeed in most Strength Challenges, but is unlikely to do the same in a Running Challenge. These points must be spent after rolling the die.

This rule only affects Challenges and may not be used in attack or reaction rolls during combat.

\subsubsection{Feat of Heroism}
A character may spend four Destiny points to perform feats of heroism that exceed the normal limits of human capacity. The key word here is heroism: Holding a house that would collapse on helpless children with the power of your muscles, disarming a bomb that would explode the city in the last minute, jumping from a height of twenty meters to hold on to a rope and climb to the villain’s helicopter, and so on.

A feat of heroism is always successful, even if the character does not have any applicable Quirk. After a feat of heroism, the master can’t continue the narrative with a “no”, but at most with a “yes, but \ldots”. “Yes, you hold the house and prevent it from falling on the orphans, but you can’t hold it more than five minutes and someone will have to get them out of there!” is a good answer to a feat of heroism.

\subsubsection{Getting Clues}
For one Destiny Point, the players can get a clue or a hint from the GM on a problem. The easiest way to do this is to have a nondescript citizen say a one-liner like “Don Tonberry hates smoke!” or “You can’t cross the Lethe river swimming” and then immediately disappear to where it came from. Characters with relevant Traits may also spend a Destiny Point to acquire relevant clues about their field of expertise.

\subsubsection{Divine Intervention}
By spending 7 Destiny Points, a character can dictate the outcome of an event or include things in the world. This divine intervention can only be invoked to save character(s) from an extremely dangerous situation or to increase the scene’s drama, but never to generate an anticlimax. Dictating that “The Shadow Lord is overcome with grief and decides to kill himself” is not an acceptable intervention but saying that “When the group is surrounded, and everyone is about to fall into the abyss, a group of giant eagles comes and grabs the characters, flying them away!” is a possible intervention. % chktex 36

If you use divine intervention when your character is dying, in a way that the character does not avoid his death, costs 4 Destiny Points instead of 7. Remember that using divine intervention by 4 points means the character will surely die at the end of intervention, and nothing that the characters can do will avoid this (except Cheat Death, below).

\subsubsection{Cheat Death}
For 10 Destiny Points, a player can restore a dead character to life or otherwise escape from death. This “resurrection” is never immediate; the character will always come back only after everyone has already given up. Remember also that Cheat Death does not mean that the character will not have consequences; he may have broken bones, health problems and mental disorders, at least temporarily. The player must decide how, exactly, the character survived (or even came back from hell itself).

\subsubsection{With Traits}
Each trait has a particular way to benefit from Destiny Point expenditure. They may be activated by spending one Destiny Point. Check the Trait list, on page~\pageref{subsec:traits}, for its description. For example, a character may spend a Destiny Point to activate his Protégé Trait and re-roll a Challenge related to saving his Protégé. 

\subsubsection{Avoiding Disadvantages}
Traits mention ways in which the character can receive penalties. If this happens, he must spend Destiny Points to avoid them. One point expenditure avoids the creation of a Challenge that the characters have to overcome in order to avoid serious consequences; two Destiny Points avoid problems causing three or more concurrent or sequential Challenges that the characters have to overcome to prevent serious problems; three Destiny Points must be spent to avoid it if the consequences of the problem caused by Trait does not need a Challenge to happen.

\begin{boco}
Optional Rule: \textbf{Solving Controversy}\pc%

When two players cannot agree on something, just have the two spend any number of Destiny Points. Whoever spends more points is right, and the other player is forced to accept his argument.
\end{boco}
\end{multicols}

\begin{center}
    \adjincludegraphics[width=0.7\textwidth]{blockwhite-cloud}
\end{center}
\clearpage