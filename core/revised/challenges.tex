\section{Challenges}\label{sec:challenges}
\begin{multicols}{2}
During a FFRPG 4th Edition adventure, the characters will face challenges created by the GM to succeed in their goals. A Challenge is a situation within the game, which has, necessarily, the following three characteristics:

\begin{enumerate}[label=\alph*]
\item \textbf{Failure and success chance.}

A Challenge must have a chance to fail and a chance to succeed. Walking on the flat and solid ground is not a Challenge, as it does not have chance of failure; walking in the air is also not a Challenge, because it has no chance of success. Walking on a rope in a cliff may be a challenge because it has chance of failure and success.

\item \textbf{Punishment for failure}

A Challenge must have a punishment for the failure, although it may be retried. This punishment for the failure may simply deny the characters the success’ reward. An attempt to open a locked door, using picks and thief’s tools is not a challenge if the character can keep trying again any number of times to get the door open, but may be a challenge if the room is filling with acid or if the lock breaks after the first attempt.

\item \textbf{Storytelling impact}

A Challenge must have a significant impact on the story. Riding a chocobo has chance of failure and success and has a punishment for failing (the fall). But in a 10-day trip riding a chocobo to journey between two cities, the way the character will ride does not have a significant enough impact on the story to become a Challenge. However, if the character runs away from a desert castle on fire being chased by bad guys in giant robots, the way he rides can indeed be a Challenge.
\end{enumerate}

\begin{center}
    \adjincludegraphics[width=0.45\textwidth]{block-engine}
\end{center}

To create a Challenge, the Game Master should indicate what is the Challenge’s Skill and what is its difficulty. The Challenge’s difficulty is its failure chance. Thus, the GM decides what is the Challenge’s failure chance, indicated by a number between 1 (one) and 99 (ninety-nine). For a character to succeed in this Challenge, his player must roll 1d100 (a 100-sided dice or two 10-sided die, one representing the tens and the other the singles) and the die roll should be greater than the established difficulty. For example, if the GM establishes 30 as the difficulty of a challenge, the player must understand that he has 30\% chance of failing, and consequently 70\% chance of success. When you fail in a Challenge, if you have the appropriate Skill, you may re-roll it. For each level in that Skill, you may re-roll the d100 once. Use the best result as your challenge roll. A player may always forfeit a Challenge. If he does, his character automatically fails. Notice that this does NOT means the character wants to fail. A character may still try very hard to achieve a forfeited Challenge but will fail nonetheless. When a player chooses to forfeit a Challenge, he may describe how his character fails.

\begin{boco}
Optional Rule: \textbf{General Acumen}\pc%

The game assumes that Stats have no direct relation with the character's capabilities, so a character with high Earth and no Strength Skill is not strong. Some groups prefer Stats with a greater impact on challenge. If you use this optional rule, add the Stat Level to the result of all Challenge rolls linked to that Stat's Skills.
\end{boco}
\end{multicols}
\clearpage