\section{Wealth}

\begin{multicols}{2}\label{sec:inv-wealth}
Gil is the currency present in most Final Fantasy games. Depending on how technologically advanced the world is, it can take many forms: gold coins, paper money, checks, planetary credit\ldots{} Regardless of Gil’s actual format, it will be an abstraction for monetary values.

Gil is a character reward as important as experience. Many Jobs take into account not only the Stat levels, but also the character’s equipment and items to measure their power. Depriving an \nameref{subsec:pjob-archer}, for example, of a suitable weapon can make the character feel extremely weak in comparison to a Mage. Imagine an \nameref{subsec:sjob-alchemist} who don’t even have an item to use? Gil is such an important reward that there is a Job dedicated to it: the \nameref{subsec:pjob-rogue}.

\subsection{Mundane Items}\label{subsec:inv-mundane}
How much does a mundane item cost? How many feet of rope, how many torches, how many sleeping bags a character has or may possess? The answer has to do directly with the economy of a Final Fantasy world: \textbf{mundane items are free}.

Everything that costs Gil refers to combat. All other items should be designed according to the context and the Challenges and Destiny system, starting at page~\pageref{ch:engine}. Mundane items are a tool for the Game Master to use in drama; they were never important in any game in the series, unless the situation said so. If you decide that your character should have a torch, they will. If it is important that your character does not have a torch, they will not. If you are unsure, use a Challenge. Maybe the Bottomless Pockets Quirk may help.

This does not mean that the characters have access to everything. For example: an aircraft is not combat-related and therefore is a ``mundane item'' is not available to the characters unless the Game Master decides so. Players can’t simply demand to own something simply because the item is mundane and therefore ``free'' as in our example aircraft. Use that desire to create an adventure --- or a small campaign arc, perhaps --- and then give them the aircraft. Use the acquisition of the aircraft to support the history and not as an accounting exercise.

That's the spirit of mundane items: something that helps you tell a story, not an exercise to prepare the best shopping list. Nothing prevents, however, NPCs to pay big bucks for a mundane item. In fact, some wealthy patrons might be willing to pay good sums of Gil for a frame, or a jewel or a miniaturized reactor that have no value for player characters. Go figure!

\subsection{Acquiring Gil}\label{subsec:inv-acquire}
You can earn Gil in many ways. As a mission’s reward; as spoils after the defeat of enemies; in chests or otherwise scattered in dungeons; by selling items; among other possibilities. During character creation, each player can spend 200 Gil in equipment and items. After this, it is recommended that for each experience point gained by the character, he should gain 8 Gil. This amount of Gil is enough to keep up with the proper equipment to their level and to buy enough healing and battle items. However, the Game Master may adjust this amount, increasing or decreasing it, to the reality of the gameplay.

Use spoils as a way randomize a bit the Gil rewards. In every combat encounter, assign one or more items as enemy’s booty. At the end of combat, after winning it, the group will roll a d100 against a set difficulty. If the die result is greater than the difficulty, they get the item(s) marked as spoils. When including spoils, multiply its value and the chance to earn it: the result is the spoils’ real value, which should be deducted from the encounter total. For example, if the Game Master decide that an encounter will give two thousand and five hundred Gil as reward and include a \tequip{Desert Ring} (worth 1,000 Gil) as spoils, with difficulty 40 (and hence 60\% chance of being obtained), they must deduct the value of six hundred Gil (60\% of 1,000). Thus, the group will receive 1,900 Gil and may earn a \tequip{Desert Ring} (or not). % chktex 36

Game Master, do not be afraid to include spoils in your game and use its randomness. In the above example, regardless of whether they got the \tequip{Desert Ring} or not, consider they received the 2,500 Gil. In the long run, the characters’ power will be the same. However, items with 100\% chance of being obtained count as normal reward, not as spoils. In the previous example, the Game Master could decide that, besides the chance to gain the \tequip{Desert Ring}, the fight would grant two Hi-Potions (150 Gil each). Thus, the combat reward would be 1,600 Gil, two Hi-Potions, and the spoils (\tequip{Desert Ring}, difficulty 40). Finally, remember that items’ sales price is half their purchase cost.

\subsection{Gil and the Rogue}\label{subsec:inv-rogue}
Early in the game, the \nameref{subsec:pjob-rogue} earns an Ability that directly influences the acquisition of Gil. \tability{Gilionaire}, \taction{Steal} and \tability{Treasure Hunter} are ways in which the \nameref{subsec:pjob-rogue} increases the amount of Gil available for the group. This translates into better and stronger items more quickly than compared with a rogueless group. However, the three skills work differently and require a different preparation from the Game Master.

\tability{Gilionaire} is the simplest of them. Whenever the group receives a Gil reward, for any reason, increase by 25\% the \nameref{subsec:pjob-rogue}’s share. For example, a group of four players defeat a monster and the Game Master says the monster’s horns are valuable and can be sold for a total of 4,000 Gil. If there is a \nameref{subsec:pjob-rogue} with this Ability in the group, their share (1,000 Gil) is increased by 25\% --- to 1,250 Gil. Thus, when selling the horns, the group collects 4,250 Gil, not 4,000. Similarly, people are willing to pay more for the same service, and the chests are inexplicably bigger. \tability{Gilionaire} is always active, and it is the least risky way to increase Gil received.

\begin{center}
    \adjincludegraphics[width=0.45\textwidth]{block-goldbag}
\end{center}

\tability{Treasure Hunter} is the second way the \nameref{subsec:pjob-rogue} can increase the group’s wealth. At the end of combat, when the group trying to gain spoils, the \nameref{subsec:pjob-rogue} can roll the die again, increasing the chance or even giving them the opportunity to gain double spoils. Compared to \tability{Gilionaire}, \tability{Treasure Hunter} is active in fewer moments because it only allows the \nameref{subsec:pjob-rogue} to improve the spoils received and does not affect other Gil gains, but can provide more significant bonuses when it works, by handing items that may be crucial when the group simply does not have the time to find a merchant in a safe place.

\taction{Steal} is the last way the Rogue has to increase the group’s wealth. This action also demands a special preparation by the Game Master. When creating a combat encounter, the GM must decide which are the Common item and the Rare item that every enemy will have. This is only required if there is a Rogue with \taction{Steal} in the group. It is not mandatory that each enemy must have both items; it may have only the Common item, only the Rare item, or even neither. Regardless, each enemy can only be stolen once. If the character tries to steal again an enemy who already suffered the effects of \taction{Steal}, it automatically fails.

The definition of which item(s) the enemy will have does not depend on the enemy's ability to use, equip or even physically carry the item; often, characters in Final Fantasy steal swords or even full armor suits from unsuspecting enemies like dragons or chocobos. Compared to \tability{Gilionaire} or \tability{Treasure Hunter}, \taction{Steal} is the most expensive and riskiest action – it costs actions in combat and is more likely to fail --- but it must be the most rewarding, especially if the Rogue obtain the Rare item. In addition, the stolen items go to the inventory and are available for immediate use, even within the combat. % chktex 36

Regardless of how the Rogue earn this extra Gil, the group shall not be punished for having a Rogue. Rogues with \tability{Gilionaire} should earn on average 10 Gil per experience point, and Rogues with the other Abilities could earn even more, depending on their rolls. The Game Master must resist the temptation to give smaller rewards to its players because ``The Rogue will steal a difference.''
\end{multicols}
\clearpage

\section{Equipment}\label{sec:inv-equip}

\begin{multicols}{2}

Equipment is the main use for Gil. Each character can equip exactly one weapon and armor, plus two accessories. In the case of weapons and armor, they are divided into categories. \tequip{Light Armor}, \tequip{Medium Armor} and \tequip{Heavy Armor} are the available armors. Weapons can be classified as \tequip{Light Swords / Knives}, \tequip{Heavy Weapons \& Shield}, \tequip{Heavy Weapons}, \tequip{Polearms}, \tequip{Claws / Gloves}, \tequip{Twin Weapons}, \tequip{Bows}, \tequip{Throwing Weapons}, \tequip{Rifle / Crossbows}, \tequip{Staffs}, \tequip{Wands}, and \tequip{Instruments}.

The character’s Abilities indicate which weapons and armor they may equip. If a character wishes to equip a weapon that they may not, it doesn’t add their Offensive Stat to any roll or attack; If a character wishes to equip an armor that they may not, they don’t consider their Defensive Stat in opponents’ rolls or attacks.

All equipment has a minimum level to be used, and most have a Gil cost. The items that do not have Gil costs are so rare that there are only a few copies of each in the world or artifacts so hard to be found that there are at most one, maybe two copies worldwide. Rare and artifact items should be given at the GM’s discretion.

\subsection{Armor}\label{subsec:inv-armor}

Armors are the main defensive equipment. Although some specific armor can empower attacks, their main function is to reduce the damage suffered. All armor pieces have two characteristics: Armor (ARM) and Magic Armor (MARM). Every time you suffer physical damage, decrease the damage by an amount equal to your ARM.\@{}Every time you suffer magical damage, decrease the damage by an amount equal to your MARM.\@{}

\begin{boco}
    For example, a character with MARM 21 suffers a magical attack dealing 46 damage. They lose 25 HP due to their magical protection. Regardless of ARM and MARM values and the damage dealt, every successful attack that deals damage causes a minimal loss of 1 HP.\@{}As it takes some time to don an armor suit, it is impossible to equip or change armor in combat.
\end{boco}

\end{multicols}

\begin{center}
    \adjincludegraphics[width=0.75\textwidth]{block-entering-building}
\end{center}
\clearpage

\subsubsection{Light Armor}

This armor is commonly used by Mages and Rune Knightss. Its main purpose is to protect against magical damage, leaving it relatively ineffective against physical damage.

\begin{tabarm}[label=inv-larm]
    % Auto-generated file, see csvtotex.py

\tabarmrow[name={Cotton Robe},mlevel={1},cost={100},arm={0},marm={3},effect={}]
\tabarmrow[name={Snow Cape},mlevel={1},cost={142},arm={0},marm={3},effect={Auto – Resist (Ice)}]
\tabarmrow[name={Mistle Robe},mlevel={1},cost={143},arm={0},marm={3},effect={Sleep Immunity}]
\tabarmrow[name={Leather Robe},mlevel={10},cost={585},arm={2},marm={10},effect={}]
\tabarmrow[name={Temple Cloth},mlevel={10},cost={866},arm={2},marm={10},effect={Auto – Resist (Shadow)}]
\tabarmrow[name={Thunder Robe},mlevel={10},cost={867},arm={2},marm={10},effect={Mute Immunity}]
\tabarmrow[name={Linen Robe},mlevel={19},cost={1552},arm={3},marm={15},effect={}]
\tabarmrow[name={Mist Silk Robe},mlevel={19},cost={2046},arm={3},marm={15},effect={Auto – Immune (Light)}]
\tabarmrow[name={Red Robe},mlevel={19},cost={2046},arm={7},marm={26},effect={SOS- Mute}]
\tabarmrow[name={Silk Robe},mlevel={28},cost={2678},arm={13},marm={36},effect={}]
\tabarmrow[name={Magician Robe},mlevel={28},cost={3797},arm={13},marm={36},effect={Add Fire level to Spells’ damage}]
\tabarmrow[name={Silver Coat},mlevel={28},cost={3787},arm={13},marm={36},effect={Auto – Resist (Crush)}]
\tabarmrow[name={Poet Robe},mlevel={37},cost={5547},arm={28},marm={65},effect={}]
\tabarmrow[name={Karate Robe},mlevel={37},cost={8391},arm={28},marm={65},effect={Decrease by 10 reactions’ difficulty}]
\tabarmrow[name={Peace Cape},mlevel={37},cost={8368},arm={28},marm={65},effect={Auto – Protect\newline{}Auto – Weaken (Physical)}]
\tabarmrow[name={Scholar Coat},mlevel={46},cost={11220},arm={41},marm={90},effect={}]
\tabarmrow[name={Aqua Robe},mlevel={46},cost={15422},arm={41},marm={90},effect={Fatal Immune}]
\tabarmrow[name={Priest's Robe},mlevel={46},cost={15424},arm={23},marm={90},effect={SOS- Wall}]
\tabarmrow[name={Tao Robe},mlevel={55},cost={Rare},arm={59},marm={127},effect={Auto–Strengthen (Mental) Auto–Strengthen (Magic)}]
\tabarmrow[name={Glutton's Coat},mlevel={55},cost={Rare},arm={59},marm={127},effect={Auto – Absorb (Bio)\newline{}Toxic Immune}]
\tabarmrow[name={Angel Robe},mlevel={55},cost={Rare},arm={59},marm={127},effect={Auto – Reraise\newline{}Gravity Immune}]
\tabarmrow[name={Element Robe},mlevel={64},cost={Artifact},arm={87},marm={179},effect={Auto – Immune (Fire, Ice, Air, Lightning, Earth and Water)}]
\tabarmrow[name={Protect Cape},mlevel={64},cost={Artifact},arm={107},marm={189},effect={Transformation Immune}]
\tabarmrow[name={Robe of Lords},mlevel={64},cost={Artifact},arm={87},marm={179},effect={Add twice Fire level to Spells’ damage; Maximum MP + 20\%}]
\end{tabarm}

\clearpage
\subsubsection{Medium Armor}

This armor type is common between many different Jobs.  It is balanced between physical and magic defenses, leaving no easily exploitable weakness.

\begin{tabarm}[label=inv-marm]
    % Auto-generated file, see csvtotex.py

\tabarmrow[name={Leather Outfit},mlevel={1},cost={84},arm={1},marm={1},effect={}]
\tabarmrow[name={Storm Jerkin},mlevel={1},cost={121},arm={1},marm={1},effect={\tfxautov{Resist}{Lightning}}]
\tabarmrow[name={Training Suit},mlevel={1},cost={121},arm={1},marm={1},effect={\tfximmunity{Blind}}]
\tabarmrow[name={Bronze Vest},mlevel={10},cost={500},arm={5},marm={5},effect={}]
\tabarmrow[name={Nomad's Tunic},mlevel={10},cost={730},arm={5},marm={5},effect={\tfxautov{Resist}{Earth}}]
\tabarmrow[name={Red Jacket},mlevel={10},cost={732},arm={5},marm={5},effect={\tfximmunity{Slow}}]
\tabarmrow[name={Chain Vest},mlevel={19},cost={1316},arm={7},marm={7},effect={}]
\tabarmrow[name={Power Sash},mlevel={19},cost={1752},arm={7},marm={7},effect={\tfximmunity{Weaken: Physical}}]
\tabarmrow[name={Survival Vest},mlevel={19},cost={1742},arm={7},marm={7},effect={\tfximmunity{Weaken: Magic}}]
\tabarmrow[name={Ring Mail},mlevel={28},cost={2269},arm={21},marm={21},effect={}]
\tabarmrow[name={Chocobo Costume},mlevel={28},cost={3201},arm={21},marm={21},effect={Add Air level your actions’ damage, except Spells and \taction{Attack}}]
\tabarmrow[name={Padded Shirt},mlevel={28},cost={3224},arm={21},marm={21},effect={\tfxautov{Resist}{Cut}}]
\tabarmrow[name={Mythril Vest},mlevel={37},cost={4675},arm={42},marm={42},effect={}]
\tabarmrow[name={Scorpion Harness},mlevel={37},cost={7082},arm={42},marm={42},effect={-10 difficulty to all physical actions}]
\tabarmrow[name={Mirage Vest},mlevel={37},cost={7147},arm={42},marm={42},effect={\tfxauto{Blink}}]
\tabarmrow[name={Brigandine},mlevel={46},cost={9563},arm={60},marm={60},effect={}]
\tabarmrow[name={Judge Coat},mlevel={46},cost={13098},arm={60},marm={60},effect={\tfximmunity{Time}}]
\tabarmrow[name={Ninja Gear},mlevel={46},cost={13055},arm={55},marm={55},effect={\tfxsos{Vanish}}]
\tabarmrow[name={Platinum Vest},mlevel={55},cost={Rare},arm={87},marm={87},effect={\tfxauto{Immune}{Earth}\newline{}\tfximmunity{Meltdown}}]
\tabarmrow[name={Behemoth Suit},mlevel={55},cost={Rare},arm={87},marm={87},effect={\tfxautov{Immune}{Light}\newline{}\tfxautov{Immune}{Shadow}}]
\tabarmrow[name={Reaper Cloak},mlevel={55},cost={Rare},arm={97},marm={97},effect={\tfxtimmunity{Fatal}}]
\tabarmrow[name={Braver Vest},mlevel={64},cost={Artifact},arm={119},marm={119},effect={\tfxauto{Strengthen: Speed}\newline{}\tfxtimmunity{Weaken} \review{Is this meant to be weaken type immunity?}}]
\tabarmrow[name={Snow Muffler},mlevel={64},cost={Artifact},arm={119},marm={119},effect={\tfxautov{Absorb}{Fire}\newline{}\tfxautov{Abosrb}{Lightning}\newline{}\tfxautov{Absorb}{Ice}}]
\tabarmrow[name={Wygar},mlevel={64},cost={Artifact},arm={129},marm={129},effect={+20\% Maximum HP}]
\end{tabarm}

\clearpage
\subsubsection{Heavy Armor}

Favoring defense against physical damage, this type of armor is popular among Warriors and Adepts.  However, it does leave the wearer more vulnerable to magical damage.

\begin{tabarm}[label=inv-harm]
    % Auto-generated file, see csvtotex.py

\tabarmrow[name={Leather Plate},mlevel={1},cost={99},arm={4},marm={1},effect={}]
\tabarmrow[name={Fire Armor},mlevel={1},cost={141},arm={4},marm={1},effect={\tfxautov{Resist}{Fire}}]
\tabarmrow[name={Cobra Cuirass},mlevel={1},cost={141},arm={4},marm={1},effect={\tfximmunity{Poison}}]
\tabarmrow[name={Bronze Armor},mlevel={10},cost={583},arm={10},marm={2},effect={}]
\tabarmrow[name={Blue Plate},mlevel={10},cost={860},arm={10},marm={2},effect={\tfxautov{Resist}{Water}}]
\tabarmrow[name={Bone Plate},mlevel={10},cost={865},arm={10},marm={2},effect={\tfximmunity{Disabled}}]
\tabarmrow[name={Plate Mail},mlevel={19},cost={1555},arm={17},marm={4},effect={}]
\tabarmrow[name={Storm Plate},mlevel={19},cost={2047},arm={17},marm={4},effect={\tfxautov{Immune}{Air}}]
\tabarmrow[name={Viking Armor},mlevel={19},cost={2060},arm={23},marm={6},effect={\tfxsos{Berserk}}]
\tabarmrow[name={Silver Mail},mlevel={28},cost={2667},arm={32},marm={11},effect={}]
\tabarmrow[name={Soldier's Armor},mlevel={28},cost={3801},arm={32},marm={11},effect={Add Earth level to \taction{Attack} damage}]
\tabarmrow[name={Carapace Mail},mlevel={28},cost={3777},arm={32},marm={11},effect={\tfxautov{Resist}{Puncture}}]
\tabarmrow[name={Mythril Armor},mlevel={37},cost={5549},arm={58},marm={25},effect={}]
\tabarmrow[name={Force Armor},mlevel={37},cost={8382},arm={58},marm={25},effect={-10 difficulty to all Spells and magical actions}]
\tabarmrow[name={Shell Mail},mlevel={37},cost={8334},arm={58},marm={25},effect={\tfxauto{Shell}}]
\tabarmrow[name={Gold Armor},mlevel={46},cost={11261},arm={81},marm={36},effect={}]
\tabarmrow[name={Aurora Mail},mlevel={46},cost={15418},arm={81},marm={36},effect={\tfxtimmunity{Seal}}]
\tabarmrow[name={Reflect Plate},mlevel={46},cost={15459},arm={81},marm={20},effect={\tfxsos{Reflect}}]
\tabarmrow[name={Platinum Armor},mlevel={55},cost={Rare},arm={114},marm={53},effect={\tfxauto{Strengthen: Physical}\newline{}\tfxauto{Strengthen: Armor}}]
\tabarmrow[name={Carabini Mail},mlevel={55},cost={Rare},arm={114},marm={53},effect={At the beginning of each round, choose one: \tvarstatus{Immune}{Cut}, \tvarstatus{Immune}{Crush}, or \tvarstatus{Immune}{Puncture}}]
\tabarmrow[name={Dragon Mail},mlevel={55},cost={Rare},arm={114},marm={53},effect={Add Earth level to \taction{Attack} damage\newline{}+10\% Maximum HP}]
\tabarmrow[name={Aegis Armor},mlevel={64},cost={Artifact},arm={161},marm={78},effect={\tfxauto{Float}\newline{}\tfxtimmunity{Mental}}]
\tabarmrow[name={Genji Armor},mlevel={64},cost={Artifact},arm={170},marm={96},effect={\tfxtresistance{All Negative}}]
\tabarmrow[name={Maximillian},mlevel={64},cost={Artifact},arm={161},marm={78},effect={\tfxauto{Immune}: \telem{Light}, \telem{Shadow}, \telem{Bio}\newline{}\tfxauto{Resist}: \telem{Cut}, \telem{Crush}, \telem{Puncture}}]
\end{tabarm}

\clearpage
\subsection{Weapons}\label{subsec:inv-weapons}

\begin{multicols}{2}
Weapons are the main offensive equipment.  Unlike armor, which are very similar, each type of weapon is quite different from others.  Each type of weapon has an Offensive Stat and a Defensive Stat.  When you use the \taction{Attack} or any other action that specifies a ``weapon attack'', you attack using the equipped weapon's Offensive and Defensive Stats unless the action says otherwise.  For example, Flamberge's (a Light Sword / Knife) damage is equal to 10x.  As its Offensive Stat is Air, the damage will be equal to ten times your Air level before any other modifiers.

In addition, the attack and damage type (physical or magical), the element (usually Cut, Puncture, or Crush), and if the attack is Ranged or Melee, all depends on the equipped weapon.

Every action that needs a weapon attack to hit and does not state the action's range (Ranged or Melee) and/or the action's elements uses the base weapon's range and element.  Actions that do state their range and/or element override the base weapon's characteristics.

Many weapons also have special effects.  All effects that modify the weapon's attack or damage only work with the \taction{Attack} action, unless the action says otherwise.  The exceptions to this rule are all effects that increase damage by some Stat level like the Colichemarde and the Soul Eater weapon ability.  Actions that key off weapon damage use the damage increased by the special effect and the Soul Eater applies to all actions made with that weapon that require a weapon attack.
\end{multicols}

\begin{center}
    \adjincludegraphics[width=0.9\textwidth]{block-shopping}
\end{center}
\clearpage
\subsubsection{Light Swords / Knives}

Fast weapons that deal \telem{Puncture}-elemental damage.  Their Offensive and Deffensive Stats are Air.  Due to their weak damage, they are largely a defensive option.  While a character is equipped with one of these weapons, they may use the reaction \taction{Parry}.  It is used when they suffer a physical attack to make a roll Air vs Air at difficulty 40.  If successful, they don't suffer the attack's effects.  These weapons are always Melee.

\begin{tabwpn}[label=inv-lsword,range=melee,type=physical,element=puncture,roll=airvair]
    % Auto-generated file, see csvtotex.py

\tabwpnrow[name={Epee},mlevel={1},cost={55},damage={2 x},effect={}]
\tabwpnrow[name={Stinger},mlevel={1},cost={78},damage={2 x},effect={\tfxtouch{Poison}}]
\tabwpnrow[name={Dream Rapier},mlevel={1},cost={90},damage={2 x},effect={\tfxtouch{Sleep}}]
\tabwpnrow[name={Silver Rapier},mlevel={10},cost={330},damage={3 x},effect={}]
\tabwpnrow[name={Scarlette},mlevel={10},cost={495},damage={3 x},effect={\tfxedamage{Fire}}]
\tabwpnrow[name={Magic Needle},mlevel={10},cost={482},damage={3 x},effect={\tfxsos{Strengthen: Magic}}]
\tabwpnrow[name={Estoc},mlevel={19},cost={750},damage={5 x},effect={}]
\tabwpnrow[name={Fleuret},mlevel={19},cost={1150},damage={5 x},effect={Add Air level to damage}]
\tabwpnrow[name={Djinn Flyssa},mlevel={19},cost={1070},damage={5 x},effect={\tfxedamage{Air}}]
\tabwpnrow[name={Mythril Rapier},mlevel={28},cost={1500},damage={8 x},effect={}]
\tabwpnrow[name={Mail Breaker},mlevel={28},cost={2250},damage={8 x},effect={\teffect{Piercing}}]
\tabwpnrow[name={Blood Rapier},mlevel={28},cost={2440},damage={8 x},effect={\teffect{HP Drain}}]
\tabwpnrow[name={Flamberge},mlevel={37},cost={3480},damage={10 x},effect={}]
\tabwpnrow[name={Colichemarde},mlevel={37},cost={5460},damage={10 x},effect={Add twice Fire level to damage}]
\tabwpnrow[name={Joyeuse},mlevel={37},cost={5130},damage={10 x},effect={\tfxstrike{Blind}}]
\tabwpnrow[name={Main Gauche},mlevel={46},cost={6160},damage={13 x},effect={}]
\tabwpnrow[name={Holy Degen},mlevel={46},cost={9000},damage={13 x},effect={\tfxedamage{Light}}]
\tabwpnrow[name={Guespire},mlevel={46},cost={9520},damage={13 x},effect={\tfxtouch{Mute}}]
\tabwpnrow[name={Epeprism},mlevel={55},cost={Rare},damage={14 x},effect={\tfxauto{Reflect}}]
\tabwpnrow[name={Dragon Fang},mlevel={55},cost={Rare},damage={14 x},effect={\tfxkiller{Dragon}\newline{}\tfxsos{Haste}}]
\tabwpnrow[name={Last Letter},mlevel={55},cost={Rare},damage={14 x},effect={\tfxtouch{Immobilize}}]
\tabwpnrow[name={Femme Fatale},mlevel={64},cost={Artifact},damage={16 x},effect={\tfxedamage{Shadow}\newline{}\tfxtouch{Death}}]
\tabwpnrow[name={Gustavian},mlevel={64},cost={Artifact},damage={16 x},effect={\tfxtouch{Disable}\newline{}-10 \taction{Parry} difficulty}]
\tabwpnrow[name={Diabolique},mlevel={64},cost={Artifact},damage={16 x},effect={\tfxauto{Haste} \newline -10 \taction{Attack} difficulty}]
\end{tabwpn}
\clearpage
\subsubsection{Heavy Weapons \& Shields}

One-handed swords, axes, hammers, maces and flails, used in conjunction with a reliable shield. Swords and axes deal Cut-elemental damage, while hammers, maces and flails deal Crush-elemental damage. It is considered a primarily defensive option as it doesn’t have high damage. Their Offensive and Defensive Stats are Earth. While you are equipped with these weapons, you can use the reaction \taction{Block}. Use when you are hit by a physical attack. Roll Earth vs Earth, difficulty 40. If successful you do not suffer the effects of the attack.

\begin{tabwpn}[label=inv-hwpnshld,range=melee,type=physical,element=*,roll=earthvearth]
    % Auto-generated file, see csvtotex.py

\tabwpnrow[name={Short Sword / Bronze Mace},mlevel={1},cost={55},damage={2 x},effect={}]
\tabwpnrow[name={Twilight Steel},mlevel={1},cost={79},damage={2 x},effect={\tfxtouch{Blind}}]
\tabwpnrow[name={Hammer of Fear},mlevel={1},cost={90},damage={2 x},effect={\tfxtouch{Weaken: Mental}}]
\tabwpnrow[name={Long Sword / Iron Mace},mlevel={10},cost={332},damage={3 x},effect={}]
\tabwpnrow[name={Ice Saber},mlevel={10},cost={498},damage={3 x},effect={\tfxedamage{Ice}}]
\tabwpnrow[name={Blue Flail},mlevel={10},cost={484},damage={3 x},effect={Add Water level to damage}]
\tabwpnrow[name={Bastard Sword / Morningstar},mlevel={19},cost={753},damage={5 x},effect={}]
\tabwpnrow[name={Mind Flail},mlevel={19},cost={1157},damage={5 x},effect={\teffect{MP Drain}}]
\tabwpnrow[name={Ancient Sword},mlevel={19},cost={1077},damage={5 x},effect={\tfxedamage{Earth}}]
\tabwpnrow[name={Mythril Sword / Mythril Mace},mlevel={28},cost={1503},damage={8 x},effect={}]
\tabwpnrow[name={Watchful},mlevel={28},cost={2268},damage={8 x},effect={\tfxkiller{Beast}}]
\tabwpnrow[name={Regal Cutlass},mlevel={28},cost={2446},damage={8 x},effect={Add Earth level to damage}]
\tabwpnrow[name={Falchion / Triple Flail},mlevel={37},cost={3483},damage={10 x},effect={}]
\tabwpnrow[name={Blood Hammer},mlevel={37},cost={5484},damage={10 x},effect={\tability{Improved Critical}}]
\tabwpnrow[name={Demon Slicer},mlevel={37},cost={5170},damage={10 x},effect={\tfxkiller{Demon}}]
\tabwpnrow[name={Scimitar / War Hammer},mlevel={46},cost={6203},damage={13 x},effect={}]
\tabwpnrow[name={Cold Steel},mlevel={46},cost={9022},damage={13 x},effect={\tfxtouch{Slow}}]
\tabwpnrow[name={Soul Saber},mlevel={46},cost={9587},damage={13 x},effect={\teffect{Ability}}]
\tabwpnrow[name={Enhancer},mlevel={55},cost={Rare},damage={14 x},effect={\tfxauto{Strengthen: Magic}\newline{}\tfxauto{Strengthen: Physical}}]
\tabwpnrow[name={Mage Killer},mlevel={55},cost={Rare},damage={14 x},effect={\tfxauto{Shell}\newline{}\teffect{Arcane Damage}}]
\tabwpnrow[name={Dancing Saber},mlevel={55},cost={Rare},damage={14 x},effect={\tfxtouch{Confuse}\newline{}Add Air level to damage}]
\tabwpnrow[name={Ragnarok},mlevel={64},cost={Artifact},damage={16 x},effect={\tfxedamage{Light}\newline{}\tfxtouch{Stop}}]
\tabwpnrow[name={Excalibur},mlevel={64},cost={Artifact},damage={16 x},effect={\tfxauto{Haste}\newline{}Add Fire level to damage}]
\tabwpnrow[name={Plague Bearer},mlevel={64},cost={Artifact},damage={16 x},effect={\tfxedamage{Bio}\newline{}\tfxstrike{Virus}\newline{}\tfxcritspell{Venom}}]
\end{tabwpn}
\clearpage
\subsubsection{Heavy Weapons}

Two-handed swords, axes, sledgehammers, pickaxes, scythes and flails that deal physical damage. Swords, axes and scythes deal Cut-elemental damage, hammers and flails deal Crush-elemental damage, while pickaxes deal Puncture-elemental damage. These weapons are the Earth-based choice with the highest damage potential. Its Offensive and Defensive Stats are Earth. This weapon type is widely used by characters who favor offense over defense.

\begin{tabwpn}[label=inv-hwpn,range=melee,type=physical,element=*,roll=earthvearth]
    % Auto-generated file, see csvtotex.py

\tabwpnrow[name={Bronze Axe / Bronze Maul},mlevel={1},cost={70},damage={3 x},effect={}]
\tabwpnrow[name={Coral Sword},mlevel={1},cost={92},damage={3 x},effect={\tfxedamage{Lightning}}]
\tabwpnrow[name={Poison Steel},mlevel={1},cost={108},damage={3 x},effect={\tfxtouch{Poison}}]
\tabwpnrow[name={Iron Axe / Iron Maul},mlevel={10},cost={412},damage={5 x},effect={}]
\tabwpnrow[name={Gishdancer},mlevel={10},cost={620},damage={5 x},effect={Add Fire level to damage}]
\tabwpnrow[name={Cosmic Axe},mlevel={10},cost={595},damage={5 x},effect={\tfxcritspell{Meteorite}}]
\tabwpnrow[name={Steel Axe / Steel Maul},mlevel={19},cost={1090},damage={7 x},effect={}]
\tabwpnrow[name={Air Pick},mlevel={19},cost={1380},damage={6 x},effect={\tfxauto{Flight}}]
\tabwpnrow[name={Demon Blade},mlevel={19},cost={1510},damage={9 x},effect={\teffect{Soul Eater}}]
\tabwpnrow[name={Mythril Axe / Mythril Pick},mlevel={28},cost={1880},damage={9 x},effect={}]
\tabwpnrow[name={Old Axe},mlevel={28},cost={2620},damage={9 x},effect={\tfxtouch{Slow}}]
\tabwpnrow[name={Arcane Buster},mlevel={28},cost={2710},damage={9 x},effect={\teffect{Arcane Damage}}]
\tabwpnrow[name={War Scythe},mlevel={37},cost={3890},damage={10 x},effect={}]
\tabwpnrow[name={Viking Axe},mlevel={37},cost={5770},damage={10 x},effect={\tfxtouch{Berserk}}]
\tabwpnrow[name={Greatsword},mlevel={37},cost={6020},damage={10 x},effect={Add Earth level to damage}]
\tabwpnrow[name={War Axe / War Maul},mlevel={46},cost={7902},damage={13 x},effect={}]
\tabwpnrow[name={Blood Axe},mlevel={46},cost={11810},damage={13 x},effect={\teffect{HP Drain}}]
\tabwpnrow[name={Golem Buster},mlevel={46},cost={9890},damage={13 x},effect={\tfxdestroyer{Construct}}]
\tabwpnrow[name={Save the Queen},mlevel={55},cost={Rare},damage={15 x},effect={Add twice Fire level to damage}]
\tabwpnrow[name={Hexenjäger},mlevel={55},cost={Rare},damage={15 x},effect={\teffect{Arcane Destruction}\newline{}\tfxcritspell{Dispel}}]
\tabwpnrow[name={Lionheart},mlevel={55},cost={Rare},damage={15 x},effect={\tfxauto{Protect}\newline{}\tfxsos{Blink}}]
\tabwpnrow[name={Juggernaut},mlevel={64},cost={Artifact},damage={19 x},effect={\tfxauto{Strengthen: Mental}\newline{}\tfxsos{Regen}}]
\tabwpnrow[name={Executioner},mlevel={64},cost={Artifact},damage={18 x},effect={\tfxedamage{Shadow}\newline{}\tfxauto{Premonition}}]
\tabwpnrow[name={Apocalypse},mlevel={64},cost={Artifact},damage={18 x},effect={\tfxedamage{Fire}\newline{}Ignores target's Armor}]
\end{tabwpn}
\clearpage

\subsubsection{Polearms}

Spears, glaives and other two-handed weapons that deal Puncture-elemental physical damage. Its Offensive Stat may be Earth or Air, and its Defensive Stat is Earth. Due to its versatility, you must choose whether to use Earth or Air as the Offensive Stat for each attack with this weapon type. They are a good option for characters that want to deal a good physical damage and need high levels in both Air and Earth, as the Warrior. 

\begin{tabwpn}[label=inv-pole,range=melee,type=physical,element=puncture,roll=earthvearth]
    % Auto-generated file, see csvtotex.py

\tabwpnrow[name={Iron Spear},mlevel={1},cost={63},damage={3 x},effect={}]
\tabwpnrow[name={Harpoon},mlevel={1},cost={85},damage={3 x},effect={Aquan Killer}]
\tabwpnrow[name={Hunter's Spear},mlevel={1},cost={99},damage={3 x},effect={Sensor}]
\tabwpnrow[name={Steel Lance},mlevel={10},cost={371},damage={4 x},effect={}]
\tabwpnrow[name={Zephyr Pike},mlevel={10},cost={558},damage={4 x},effect={Air-elemental damage}]
\tabwpnrow[name={Shaman's Lance},mlevel={10},cost={539},damage={4 x},effect={Add Fire level to damage}]
\tabwpnrow[name={Mythril Pike},mlevel={19},cost={920},damage={6 x},effect={}]
\tabwpnrow[name={Halberd},mlevel={19},cost={1265},damage={6 x},effect={SOS- Haste}]
\tabwpnrow[name={Web Lance},mlevel={19},cost={1290},damage={6 x},effect={Slow Touch}]
\tabwpnrow[name={Gold Lance},mlevel={28},cost={1690},damage={9 x},effect={}]
\tabwpnrow[name={Manhunter},mlevel={28},cost={2435},damage={9 x},effect={Humanoid Killer}]
\tabwpnrow[name={Stout Spear},mlevel={28},cost={2575},damage={9 x},effect={Mute Touch}]
\tabwpnrow[name={Partisan},mlevel={37},cost={3685},damage={11 x},effect={}]
\tabwpnrow[name={Grey Lance},mlevel={37},cost={5615},damage={11 x},effect={Auto – Strengthen (Magic)}]
\tabwpnrow[name={Viper Halberd},mlevel={37},cost={5575},damage={11 x},effect={Piercing}]
\tabwpnrow[name={Glaive},mlevel={46},cost={7031},damage={14 x},effect={}]
\tabwpnrow[name={Guisarme},mlevel={46},cost={10405},damage={15 x},effect={}]
\tabwpnrow[name={Stoic Lance},mlevel={46},cost={9705},damage={14 x},effect={Disable Touch}]
\tabwpnrow[name={Berserker Spear},mlevel={55},cost={Rare},damage={16 x},effect={Add twice Water level to damage}]
\tabwpnrow[name={Imp Halberd},mlevel={55},cost={Rare},damage={16 x},effect={Toad Touch\newline{}Water-elemental damage}]
\tabwpnrow[name={Kain's Lance},mlevel={55},cost={Rare},damage={16 x},effect={Auto – Blink\newline{}Auto – Strengthen (Armor)}]
\tabwpnrow[name={Highwind},mlevel={64},cost={Artifact},damage={18 x},effect={Improved Critical\newline{}MP Drain}]
\tabwpnrow[name={Thanatos Lance},mlevel={64},cost={Artifact},damage={18 x},effect={Death Touch\newline{}Bio-elemental damage}]
\tabwpnrow[name={Gungnir},mlevel={64},cost={Artifact},damage={20 x},effect={Sleep Strike\newline{}Soul Eater}]
\end{tabwpn}
\clearpage

\subsubsection{Claws / Gloves}

Claws and gloves are Monks’ signature weapon, but almost any character can fight well with them. They deal physical damage, with Gloves dealing Crush-elemental and Claws dealing Cut-elemental damage. Its Offensive Stat is Earth, and its Defensive Stat is Air. Its low cost and ease of use make this weapon a good option for characters that don’t want to worry too much about weapons, and its broad spectrum of effects mean that any character can benefit if you choose the right Claw or Glove. 

\begin{tabwpn}[label=inv-claws,range=melee,type=physical,element=crush,roll=earthvair]
    % Auto-generated file, see csvtotex.py

\tabwpnrow[name={Leather Glove / Bronze Claws},mlevel={1},cost={55},damage={2 x},effect={}]
\tabwpnrow[name={Cursed Claws},mlevel={1},cost={78},damage={2 x},effect={\tfxtouch{Curse}}]
\tabwpnrow[name={Sonar},mlevel={1},cost={90},damage={2 x},effect={\teffect{Sensor}}]
\tabwpnrow[name={Metal Knuckle / Iron Claws},mlevel={10},cost={331},damage={3 x},effect={}]
\tabwpnrow[name={Dusk Knuckle},mlevel={10},cost={497},damage={3 x},effect={\tfxtouch{Poison}}]
\tabwpnrow[name={Mirage Claws},mlevel={10},cost={483},damage={3 x},effect={\tfxedamage{Earth}}]
\tabwpnrow[name={Mythril Glove / Mythril Claw},mlevel={19},cost={749},damage={5 x},effect={}]
\tabwpnrow[name={Scissor Fangs},mlevel={19},cost={1144},damage={5 x},effect={\teffect{HP Drain}}]
\tabwpnrow[name={Magic Glove},mlevel={19},cost={1070},damage={5 x},effect={\tfxtouch{Sleep}}]
\tabwpnrow[name={Gold Glove / Hell Claws},mlevel={28},cost={1497},damage={7 x},effect={}]
\tabwpnrow[name={Meteo Knuckle},mlevel={28},cost={2239},damage={7 x},effect={\tfxcritspell{Meteorite}}]
\tabwpnrow[name={Avenger},mlevel={28},cost={2447},damage={7 x},effect={\tfxtouch{Slow}}]
\tabwpnrow[name={Tiger Fangs /Power Knuckle},mlevel={37},cost={3479},damage={8 x},effect={}]
\tabwpnrow[name={Prism Claws},mlevel={37},cost={5485},damage={8 x},effect={Add Air level to damage}]
\tabwpnrow[name={The Reaper},mlevel={37},cost={5136},damage={8 x},effect={\tfxedamage{Shadow}\newline{}\tfxtouch{Condemn}}]
\tabwpnrow[name={Kaiser Claws /Kaiser Knuckle},mlevel={46},cost={6174},damage={10 x},effect={}]
\tabwpnrow[name={Ironside},mlevel={46},cost={9001},damage={10 x},effect={\tfxauto{Strengthen: Mental}}]
\tabwpnrow[name={War Monger},mlevel={46},cost={9525},damage={10 x},effect={\tfxtouch{Disable}}]
\tabwpnrow[name={Overload},mlevel={55},cost={Rare},damage={12 x},effect={\tfxafocus{Pain}\newline{}\tfxedamage{Bio}}]
\tabwpnrow[name={Devastator},mlevel={55},cost={Rare},damage={12 x},effect={\tfxedamage{Fire}\newline{}\tfxtouch{Meltdown}}]
\tabwpnrow[name={Colossus},mlevel={55},cost={Rare},damage={12 x},effect={\tfxedamage{Earth}\newline{}\tfxtouch{Stone}}]
\tabwpnrow[name={Godhand},mlevel={64},cost={Artifact},damage={14 x},effect={\tfxafocus{Hex}\newline{}\tfxkiller{Demon}}]
\tabwpnrow[name={Tempest Claws},mlevel={64},cost={Artifact},damage={14 x},effect={\tfxauto{Haste}\newline{}\teffect{Piercing}}]
\tabwpnrow[name={Infinity},mlevel={64},cost={Artifact},damage={14 x},effect={\teffect{Improved Critical}\newline{}\teffect{Triple Critical}}]
\end{tabwpn}
\clearpage
\subsection{Accessories}\label{subsec:inv-accessories}

\begin{multicols}{2}
Each character can equip up to two accessories at the same time.  However, the equipment's magic interferes with each other.  Thus, if two accessories or an accessory and another piece of equipment including a weapon or armor would have the same effect, they do not stack.  Only the most powerful item of this type will affect you.  Similarly if a character equips three or more accessories, their effects cancel each other and the character will not receive the benefits of any of them.  For example, if a character equips the Wyglar armor (+ 20\% Max HP) and the Orrachea Armlet (+ 10\% Max HP), the HP increase is only 20\%.

Morever, if two distinct pieces of equipment impart opposite effects only the negative effect will be gained.  For example, if a character is equipped with an accessory granting \tstatus{Mute} immunity and an armor with SOS-\tstatus{Mute}, that character will be immune to the \tstatus{Mute} status until their HP is below 25\%, but once their HP drops below 25\% will suffer the \tstatus{Mute} and lose their immunity.

Although it may seem simple, equipping an accessory requires the character to spend at least a few minutes focusing on its magic.  Thus, it is impossible to equip or change accessories in the middle of a combat.
\end{multicols}

\begin{tabacc}[label=inv-acc1]
    % Auto-generated file, see csvtotex.py

\tabaccrow[name={Nishijin Belt},mlevel={1},cost={500},effect={\tfximmunity{Sleep}}]
\tabaccrow[name={Silver Spectacles},mlevel={1},cost={450},effect={\tfximmunity{Blind}}]
\tabaccrow[name={Star Pendant},mlevel={1},cost={600},effect={\tfximmunity{Poison}}]
\tabaccrow[name={Aqua Ring},mlevel={10},cost={1000},effect={\tfxautov{Resist}{Water}}]
\tabaccrow[name={Barrier Ring},mlevel={10},cost={800},effect={\tfxsos{Shell}}]
\tabaccrow[name={Desert Ring},mlevel={10},cost={1000},effect={\tfxautov{Resist}{Earth}}]
\tabaccrow[name={Guard Ring},mlevel={10},cost={800},effect={\tfxsos{Protect}}]
\tabaccrow[name={Jackboots},mlevel={10},cost={600},effect={\tfximmunity{Immobilize}}]
\tabaccrow[name={Sash},mlevel={10},cost={600},effect={\tfximmunity{Slow}}]
\tabaccrow[name={Angel Wings},mlevel={18},cost={1800},effect={\tfxauto{Float}}]
\tabaccrow[name={Defense Ring},mlevel={18},cost={1500},effect={\tfximmunity{Sleep}\newline{}\tfximmunity{Condemn}}]
\tabaccrow[name={Fairy Ring},mlevel={18},cost={1250},effect={\tfximmunity{Blind}\newline{}\tfximmunity{Poison}}]
\tabaccrow[name={Gold Choker},mlevel={18},cost={1250},effect={\tfxautov{Resist}{Air}}]
\tabaccrow[name={Leather Gorget},mlevel={18},cost={1400},effect={\tfxsos{Strengthen: Magic}}]
\tabaccrow[name={Princess Ring},mlevel={18},cost={1600},effect={\tfxsos{Shell}\newline{}\tfxsos{Protect}}]
\tabaccrow[name={Star Bangle},mlevel={18},cost={1800},effect={\tfxsos{Regen}}]
\tabaccrow[name={Steel Gorget},mlevel={18},cost={1400},effect={\tfxsos{Strengthen: Pyhsical}}]
\tabaccrow[name={Bead Brooch},mlevel={25},cost={3250},effect={\tfximmunity{Blind}\newline{}\tfximmunity{Mute}}]
\tabaccrow[name={Black Belt},mlevel={25},cost={1600},effect={\tfximmunity{Disable}}]
\tabaccrow[name={Bowline Sash},mlevel={25},cost={2000},effect={\tfximmunity{Confuse}}]
\tabaccrow[name={Coral Ring},mlevel={25},cost={2500},effect={tfxautov{Immune}{Lightning}}]
\tabaccrow[name={Crystal Ball},mlevel={25},cost={3400},effect={\tfxauto{Premonition}}]
\tabaccrow[name={Echo Bangle},mlevel={25},cost={1600},effect={\tfximmunity{Mute}}]
\tabaccrow[name={Holy Mirror},mlevel={25},cost={3400},effect={\tfxauto{Blink}}]
\tabaccrow[name={Magic Bangle},mlevel={25},cost={2700},effect={+10\% max MP}]
\tabaccrow[name={Orrachea Armlet},mlevel={25},cost={2700},effect={+10\% max HP}]
\tabaccrow[name={Protect Ring},mlevel={25},cost={3250},effect={\tfxauto{Protect}}]
\tabaccrow[name={Reflect Ring},mlevel={25},cost={3500},effect={\tfxauto{Reflect}}]
\tabaccrow[name={Ring Shell},mlevel={25},cost={3250},effect={\tfxauto{Shell}}]
\tabaccrow[name={Water Ring},mlevel={25},cost={2500},effect={\tfxautov{Immune}{Water}}]
\tabaccrow[name={Amulet},mlevel={32},cost={4400},effect={\tfximmunity{Blind}\newline{}\tfximmunity{Poison}\newline{}\tfximmunity{Zombie}}]
\tabaccrow[name={Blizzard Ring},mlevel={32},cost={3300},effect={\tfxautov{Immune}{Ice}}]
\tabaccrow[name={Fire Ring},mlevel={32},cost={3300},effect={\tfxautov{Immune}{Fire}}]
\tabaccrow[name={Gravity Ring},mlevel={32},cost={3600},effect={\tfximmunity{Gravity}}]
\tabaccrow[name={Magic Ring},mlevel={32},cost={4400},effect={\tfximmunity{Mute}\newline{}\tfximmunity{Berserk}}]
\tabaccrow[name={Phantom Ring},mlevel={32},cost={5200},effect={\tfximmunity{Weaken}}]
\tabaccrow[name={Rebirth Ring},mlevel={32},cost={5500},effect={\tfxauto{Reraise}}]
\tabaccrow[name={Wing Boots},mlevel={32},cost={4700},effect={\tfxauto{Flight}}]
\tabaccrow[name={Cerulean Ring},mlevel={39},cost={5400},effect={\tfxautov{Absorb}{Water}}]
\tabaccrow[name={Flame Ring},mlevel={39},cost={8000},effect={\tfxautov{Absorb}{Fire}}]
\tabaccrow[name={Ice Ring},mlevel={39},cost={8000},effect={\tfxautov{Absorb}{Ice}}]
\tabaccrow[name={Jade Armlet},mlevel={39},cost={7600},effect={\tfximmunity{Stone}\newline{}\tfximmunity{Slow}}]
\tabaccrow[name={Ochre Ring},mlevel={39},cost={8000},effect={\tfxautov{Absorb}{Lightning}}]
\tabaccrow[name={Force Belt},mlevel={46},cost={10400},effect={+10\% max HP and MP}]
\tabaccrow[name={Gauss Buckle},mlevel={46},cost={12000},effect={\tfxautov{Immunity}{Bio}\newline{}\tfximmunity{Poison}\newline{}\tfximmunity{Virus}}]
\tabaccrow[name={Guard Bracelet},mlevel={46},cost={12000},effect={\tfxauto{Protect}\newline{}\tfxauto{Shell}}]
\tabaccrow[name={Jeweled Ring},mlevel={46},cost={10000},effect={\tfximmunity{Blind}\newline{}\tfximmunity{Sleep}\newline{}\tfximmunity{Stone}}]
\tabaccrow[name={Rosetta Ring},mlevel={46},cost={12000},effect={\tfxautov{Immune}{Fire}\newline{}\tfximmunity{Condemn}}]
\tabaccrow[name={Rubber Boots},mlevel={46},cost={10000},effect={\tfxautov{Immune}{Lightning}\newline{}\tfximmunity{Stop}}]
\tabaccrow[name={Scarab},mlevel={46},cost={9600},effect={\tfximmunity{Immobilize}\newline{}\tfximmunity{Disable}\newline{}\tfximmunity{Toad}}]
\tabaccrow[name={Snow Ring},mlevel={46},cost={12000},effect={\tfxautov{Immune}{Ice}\newline{}\tfximmunity{Berserk}}]
\tabaccrow[name={Star Armlet},mlevel={46},cost={9600},effect={\tfximmunity{Slow}\newline{}\tfximmunity{Stop}}]
\tabaccrow[name={Gold Hairpin},mlevel={53},cost={15000},effect={Your Spells cost 25\% less MP}]
\tabaccrow[name={Japa Mala},mlevel={53},cost={13000},effect={\tfxautov{Immune}{Bio}\newline{}\tfximmunity{Poison}\newline{}\tfximmunity{Virus}\newline{}\tfximmunity{Zombie}}]
\tabaccrow[name={Nu Khai Armlet},mlevel={53},cost={13000},effect={\tfxautov{Resist}{Shadow}\newline{}\tfximmunity{Confuse}\newline{}\tfximmunity{Charm}}]
\tabaccrow[name={Peach Ring},mlevel={53},cost={15000},effect={\tfximmunity{Berserk}\newline{}\tfximmunity{Confuse}\newline{}\tfximmunity{Charm}}]
\tabaccrow[name={Poison Ring},mlevel={53},cost={13000},effect={\tfxautov{Immune}{Bio}\newline{}\tfximmunity{Poison}\newline{}\tfximmunity{Virus}}]
\tabaccrow[name={Safety Bit},mlevel={53},cost={20000},effect={\tfxtimmunity{Fatal}}]
\tabaccrow[name={White Cape},mlevel={53},cost={13500},effect={\tfxtimmunity{Transformation}}]
\tabaccrow[name={Aegis Ring},mlevel={60},cost={Rare},effect={\tfxautov{Resist}{All but cut/puncture/crush}}]
\tabaccrow[name={Angel Ring},mlevel={60},cost={Rare},effect={\tfxautov{Immune}{Shadow}\newline{}\tfxauto{Reraise}\newline{}\tfxtimmunity{Fatal}}]
\tabaccrow[name={Berserker Ring},mlevel={60},cost={Rare},effect={\tfxautov{Absorb}{Fire}\newline{}\tfxautov{Immune}{Light}\newline{}\tfxauto{Berserk}}]
\tabaccrow[name={Hermes Sandals},mlevel={60},cost={Rare},effect={\tfxauto{Haste}}]
\tabaccrow[name={Mindu Jewel},mlevel={60},cost={Rare},effect={\tfxauto{Blink}\newline{}\tfximmunity{Blind, Toad, Poison, Mute, Confuse, Stone}}]
\tabaccrow[name={Power Belt},mlevel={60},cost={Rare},effect={+25\% max HP}]
\tabaccrow[name={Sorcery Bangle},mlevel={60},cost={Rare},effect={+25\% max MP}]
\tabaccrow[name={Crystal Orb},mlevel={67},cost={Artifact},effect={+50\% max MP}]
\tabaccrow[name={Economizer},mlevel={67},cost={Artifact},effect={-50\% MP cost to all Spells}]
\tabaccrow[name={Invisibility Cloak},mlevel={67},cost={Artifact},effect={\tfxauto{Vanish}}]
\tabaccrow[name={Muscle Belt},mlevel={67},cost={Artifact},effect={+50\% max HP}]
\tabaccrow[name={Ribbon},mlevel={67},cost={Artifact},effect={\tfximmunity{ALL}}]
\end{tabacc}

\clearpage
\section{Inventory}\label{sec:inv-inventory}

\begin{multicols}{2}
All items the group possess who are not equipped on a character are part of the Inventory. In addition to spare equipment, the Inventory contain consumables, such as potions, elixirs and battle items. The Inventory is a group shared resource, such as Destiny points. Thus, if the group has a potion in the Inventory, any character in the group may use the \taction{Draw} action to retrieve it. Consumables are spent after one use.

Every group should devote a part of the obtained Gil to purchase consumables for the Inventory. Even for groups capable of emulating all consumable effects through Spells and Abilities, you never know when the White Mage will drop to 0 HP and the Monk will need to use a Phoenix Down.

Not having an Inventory suitable to the challenges ahead is a sure way to shorten the group's lifespan. Having the right items to use when the occasion requires it can dramatically increase the group’s combat ability. Another advantage in favor of consumables is that they do not have minimum levels to be used, allowing a group to gain powers beyond the usual in dire circumstances.

\begin{center}
    \adjincludegraphics[width=0.45\textwidth]{block-pumpkinsale}
\end{center}

\begin{boco}
Optional Rule: \textbf{Individual Inventory} \pc%

Some groups do not like the shared Inventory rule. In that case, you may play with each character having its own individual items. If you use this rule, any character may use \taction{Draw} to draw an item from his own inventory and give to another player with a single action, but the player who received the item still must use the \taction{Item} action to use the consumable or equip the weapon. \pw%

Optional Rule: \textbf{Carrying Capacity} \pc%

By default, there is no limit about how many items a character or a group may carry. This mimics the videogames and allows for quicker play. However, if you do want to enforce carrying limits when using the aforementioned Individual Inventory rule, treat the maximum number of consumables a character may carry as two times his Earth level. Alchemists use their highest Stat level instead of Earth to calculate how many items they can carry. \pc%

If you want to enforce the Carrying Capacity rule with the default Shared Inventory option, consider the group's total carrying limit as the sum of each character's individual carrying capacity. \pc%

Characters carrying more items than their carrying capacity suffer increased difficulty in all their attacks, reactions and physical Challenges.

\end{boco}

\end{multicols}

\clearpage
\subsection{Potions and Healing Items}\label{subsec:potions}

\begin{tabitem}[label=inv-potions]
    \tabitemrow[name=Tonic,cost=25,effect=Heals 25 HP to a single target]
\end{tabitem}

\clearpage
\subsection{Battle Items}\label{subsec:battleitems}

\begin{tabitem}[label=inv-battles]
    % Auto-generated file, see csvtotex.py

\tabitemrow[name={Arctic Wind},cost={35},effect={\tfxcast{Blizzard}}]
\tabitemrow[name={Bomb Fragment},cost={35},effect={\tfxcast{Fire}}]
\tabitemrow[name={Electro Marble},cost={35},effect={\tfxcast{Thunder}}]
\tabitemrow[name={Bird Feather},cost={35},effect={\tfxcast{Aero}}]
\tabitemrow[name={Smoke Bomb},cost={55},effect={\tfxcast{Escape}, but only in battle}]
\tabitemrow[name={Graviball},cost={130},effect={\tfxcast{Gravity}}]
\tabitemrow[name={Fish Scale},cost={200},effect={\tfxcast{Water}}]
\tabitemrow[name={Warp Stone},cost={275},effect={\tfxcast{Teleport}, but only in battle}]
\tabitemrow[name={Antarctic Wind},cost={325},effect={\tfxcast{Blizzara}}]
\tabitemrow[name={Bomb Core},cost={325},effect={\tfxcast{Fira}}]
\tabitemrow[name={Lightning Marble},cost={325},effect={\tfxcast{Thundara}}]
\tabitemrow[name={Shear Feather},cost={325},effect={\tfxcast{Aera}}]
\tabitemrow[name={Zombie Powder},cost={360},effect={\tfxcast{Zombie}}]
\tabitemrow[name={Healing Spring},cost={425},effect={\tfxcast{Regen}}]
\tabitemrow[name={Light Curtain},cost={475},effect={\tfxcast{Protect}}]
\tabitemrow[name={Lunar Curtain},cost={475},effect={\tfxcast{Shell}}]
\tabitemrow[name={Fish Fin},cost={525},effect={\tfxcast{Waterga}}]
\tabitemrow[name={Vampire Fang},cost={550},effect={\tfxcast{Drain}}]
\tabitemrow[name={Stardust},cost={575},effect={\tfxcast{Comet}}]
\tabitemrow[name={Mute Mask},cost={580},effect={\tfxcast{Silence}}]
\tabitemrow[name={T / S Bomb},cost={625},effect={\tfxcast{Demi}}]
\tabitemrow[name={Speed Drink},cost={640},effect={\tfxcast{Haste}}]
\tabitemrow[name={Deadly Waste},cost={660},effect={\tfxcast{Bio}}]
\tabitemrow[name={Earth Drum},cost={680},effect={\tfxcast{Quake}}]
\tabitemrow[name={Fire Gem},cost={750},effect={\tfxcast{Firaga}}]
\tabitemrow[name={Ice Gem},cost={750},effect={\tfxcast{Blizzaga}}]
\tabitemrow[name={Lightning Gem},cost={750},effect={\tfxcast{Thundaga}}]
\tabitemrow[name={Windmill},cost={750},effect={\tfxcast{Aeraga}}]
\tabitemrow[name={Light Hammer},cost={750},effect={\tfxcast{Banishga}}]
\tabitemrow[name={Silver Hourglass},cost={800},effect={\tfxcast{Slowga}}]
\tabitemrow[name={Lunar Veil},cost={825},effect={\tfxcast{Reflect}}]
\tabitemrow[name={Light Veil},cost={850},effect={\tfxcast{Wall}}]
\tabitemrow[name={Malboro Tentacle},cost={880},effect={\tfxcast{Venom}}]
\tabitemrow[name={Ghost Hand},cost={950},effect={\tfxcast{Osmosis}}]
\tabitemrow[name={Blue Stone},cost={1100},effect={\tfxcast{Storm}}]
\tabitemrow[name={Brown Stone},cost={1100},effect={\tfxcast{Break}}]
\tabitemrow[name={Black Stone},cost={1100},effect={\tfxcast{Scathe}}]
\tabitemrow[name={Basilisk Claw},cost={1200},effect={\tfxcast{Stone}}]
\tabitemrow[name={Candle of Life},cost={1250},effect={\tfxcast{Death}}]
\tabitemrow[name={Shadow Gem},cost={1275},effect={\tfxcast{Quarter}}]
\tabitemrow[name={Purifying Salt},cost={1300},effect={\tfxcast{Dispel}}]
\tabitemrow[name={War Gong},cost={1325},effect={\tfxcast{Berserk}}]
\tabitemrow[name={Loco Weed},cost={1350},effect={\tfxcast{Confuse}}]
\tabitemrow[name={Green Stone},cost={1400},effect={\tfxcast{Virus}}]
\tabitemrow[name={Impaler},cost={1425},effect={\tfxcast{Toad}}]
\tabitemrow[name={Soul Spring},cost={1450},effect={\tfxcast{Syphon}}]
\tabitemrow[name={White Stone},cost={1600},effect={\tfxcast{Freeze}}]
\tabitemrow[name={Red Stone},cost={1600},effect={\tfxcast{Meltdown}}]
\tabitemrow[name={Yellow Stone},cost={1600},effect={\tfxcast{Overcharge}}]
\tabitemrow[name={Star Curtain},cost={1800},effect={\tfxcast{Shellga}}]
\tabitemrow[name={Adamant Shard},cost={1800},effect={\tfxcast{Protectga}}]
\tabitemrow[name={Dark Matter},cost={2000},effect={\tfxcast{Stop}}]
\tabitemrow[name={Impossible Mirror},cost={Rare},effect={\tfxcast{Wall} . While this \tstatus{Wall} lasts, reflect the effects of all physical actions to user}]
\tabitemrow[name={Hero Drink},cost={Rare},effect={Casts all of \tspell{Magic Up}, \tspell{Armor Up}, \tspell{Mental Up}, and \tspell{Power Up} on the same target}]
\tabitemrow[name={Golden Hourglass},cost={Rare},effect={\tfxcast{Old}}]
\tabitemrow[name={Meteor Stone},cost={Rare},effect={\tfxcast{Meteor}}]


\end{tabitem}
\vfill
\begin{center}
    \adjincludegraphics[width=0.75\textwidth]{block-market}
\end{center}

\clearpage
\section{Equipment Effects}\label{sec:equipeffects}

\begin{table}[h]
    \begin{tabular}{p{0.3\textwidth}<{\arraybackslash\dotfill}@{}>{\raggedleft\arraybackslash\dotfill}p{0.6\textwidth}}
        Arcane Damage & 50\% damage is dealt to MP in addition to HP \\
        Arcane Destruction & Full damage is dealt to the MP in addition to the HP \\
        Arcane Focus (Spell) & You can use this weapon to cast the listed spell, spending MP as usual.  You do not need to know the spell. \\
    \end{tabular}
\end{table}
