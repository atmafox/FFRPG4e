\label{ch:inventory}

\section{Wealth}

\begin{multicols}{2}
\label{sec:inv-wealth}
Gil is the currency present in most Final Fantasy games. Depending on how technologically advanced the world is, it can take many forms: gold coins, paper money, checks, planetary credit\ldots{} Regardless of Gil’s actual format, it will be an abstraction for monetary values.

Gil is a character reward as important as experience. Many Jobs take into account not only the Stat levels, but also the character’s equipment and items to measure their power. Depriving an \nameref{subsec:pjob-archer}, for example, of a suitable weapon can make the character feel extremely weak in comparison to a Mage. Imagine an \nameref{subsec:sjob-alchemist} who don’t even have an item to use? Gil is such an important reward that there is a Job dedicated to it: the \nameref{subsec:pjob-rogue}.

\subsection{Mundane Items}
\label{subsec:inv-mundane}
How much does a mundane item cost? How many feet of rope, how many torches, how many sleeping bags a character has or may possess? The answer has to do directly with the economy of a Final Fantasy world: \textbf{mundane items are free}.

Everything that costs Gil refers to combat. All other items should be designed according to the context and the Challenges and Destiny system, starting at page \pageref{ch:engine}. Mundane items are a tool for the Game Master to use in drama; they were never important in any game in the series, unless the situation said so. If you decide that your character should have a torch, they will. If it is important that your character does not have a torch, they will not. If you are unsure, use a Challenge. Maybe the Bottomless Pockets Quirk may help.

This does not mean that the characters have access to everything. For example: an aircraft is not combat-related and therefore is a "mundane item" is not available to the characters unless the Game Master decides so. Players can’t simply demand to own something simply because the item is mundane and therefore "free" as in our example aircraft. Use that desire to create an adventure - or a small campaign arc, perhaps - and then give them the aircraft. Use the acquisition of the aircraft to support the history and not as an accounting exercise.

That's the spirit of mundane items: something that helps you tell a story, not an exercise to prepare the best shopping list. Nothing prevents, however, NPCs to pay big bucks for a mundane item. In fact, some wealthy patrons might be willing to pay good sums of Gil for a frame, or a jewel or a miniaturized reactor that have no value for player characters. Go figure!

\subsection{Acquiring Gil}
\label{subsec:inv-acquire}
You can earn Gil in many ways. As a mission’s reward; as spoils after the defeat of enemies; in chests or otherwise scattered in dungeons; by selling items; among other possibilities. During character creation, each player can spend 200 Gil in equipment and items. After this, it is recommended that for each experience point gained by the character, he should gain 8 Gil. This amount of Gil is enough to keep up with the proper equipment to their level and to buy enough healing and battle items. However, the Game Master may adjust this amount, increasing or decreasing it, to the reality of the gameplay.

Use spoils as a way randomize a bit the Gil rewards. In every combat encounter, assign one or more items as enemy’s booty. At the end of combat, after winning it, the group will roll a d100 against a set difficulty. If the die result is greater than the difficulty, they get the item(s) marked as spoils. When including spoils, multiply its value and the chance to earn it: the result is the spoils’ real value, which should be deducted from the encounter total. For example, if the Game Master decide that an encounter will give two thousand and five hundred Gil as reward and include a \tequip{Desert Ring} (worth 1,000 Gil) as spoils, with difficulty 40 (and hence 60\% chance of being obtained), they must deduct the value of six hundred Gil (60\% of 1,000). Thus, the group will receive 1,900 Gil and may earn a \tequip{Desert Ring} (or not).

Game Master, do not be afraid to include spoils in your game and use its randomness. In the above example, regardless of whether they got the \tequip{Desert Ring} or not, consider they received the 2,500 Gil. In the long run, the characters’ power will be the same. However, items with 100\% chance of being obtained count as normal reward, not as spoils. In the previous example, the Game Master could decide that, besides the chance to gain the \tequip{Desert Ring}, the fight would grant two Hi-Potions (150 Gil each). Thus, the combat reward would be 1,600 Gil, two Hi-Potions, and the spoils (\tequip{Desert Ring}, difficulty 40). Finally, remember that items’ sales price is half their purchase cost.

\subsection{Gil and the Rogue}
\label{subsec:inv-rogue}
Early in the game, the \nameref{subsec:pjob-rogue} earns an Ability that directly influences the acquisition of Gil. \tability{Gilionaire}, \taction{Steal} and \tability{Treasure Hunter} are ways in which the \nameref{subsec:pjob-rogue} increases the amount of Gil available for the group. This translates into better and stronger items more quickly than compared with a rogueless group. However, the three skills work differently and require a different preparation from the Game Master.

\tability{Gilionaire} is the simplest of them. Whenever the group receives a Gil reward, for any reason, increase by 25\% the \nameref{subsec:pjob-rogue}’s share. For example, a group of four players defeat a monster and the Game Master says the monster’s horns are valuable and can be sold for a total of 4,000 Gil. If there is a \nameref{subsec:pjob-rogue} with this Ability in the group, their share (1,000 Gil) is increased by 25\% - to 1,250 Gil. Thus, when selling the horns, the group collects 4,250 Gil, not 4,000. Similarly, people are willing to pay more for the same service, and the chests are inexplicably bigger. \tability{Gilionaire} is always active, and it is the least risky way to increase Gil received.

\begin{center}
    \adjincludegraphics[width=0.45\textwidth]{block-goldbag}
\end{center}

\tability{Treasure Hunter} is the second way the \nameref{subsec:pjob-rogue} can increase the group’s wealth. At the end of combat, when the group trying to gain spoils, the \nameref{subsec:pjob-rogue} can roll the die again, increasing the chance or even giving them the opportunity to gain double spoils. Compared to \tability{Gilionaire}, \tability{Treasure Hunter} is active in fewer moments because it only allows the \nameref{subsec:pjob-rogue} to improve the spoils received and does not affect other Gil gains, but can provide more significant bonuses when it works, by handing items that may be crucial when the group simply does not have the time to find a merchant in a safe place.

\taction{Steal} is the last way the Rogue has to increase the group’s wealth. This action also demands a special preparation by the Game Master. When creating a combat encounter, the GM must decide which are the Common item and the Rare item that every enemy will have. This is only required if there is a Rogue with \taction{Steal} in the group. It is not mandatory that each enemy must have both items; it may have only the Common item, only the Rare item, or even neither. Regardless, each enemy can only be stolen once. If the character tries to steal again an enemy who already suffered the effects of \taction{Steal}, it automatically fails.

The definition of which item(s) the enemy will have does not depend on the enemy's ability to use, equip or even physically carry the item; often, characters in Final Fantasy steal swords or even full armor suits from unsuspecting enemies like dragons or chocobos. Compared to \tability{Gilionaire} or \tability{Treasure Hunter}, \taction{Steal} is the most expensive and riskiest action – it costs actions in combat and is more likely to fail - but it must be the most rewarding, especially if the Rogue obtain the Rare item. In addition, the stolen items go to the inventory and are available for immediate use, even within the combat.

Regardless of how the Rogue earn this extra Gil, the group shall not be punished for having a Rogue. Rogues with \tability{Gilionaire} should earn on average 10 Gil per experience point, and Rogues with the other Abilities could earn even more, depending on their rolls. The Game Master must resist the temptation to give smaller rewards to its players because "The Rogue will steal a difference."
\end{multicols}
\clearpage

\section{Equipment}
\label{sec:inv-equip}

\begin{multicols}{2}

Equipment is the main use for Gil. Each character can equip exactly one weapon and armor, plus two accessories. In the case of weapons and armor, they are divided into categories. \tequip{Light Armor}, \tequip{Medium Armor} and \tequip{Heavy Armor} are the available armors. Weapons can be classified as \tequip{Light Swords / Knives}, \tequip{Heavy Weapons \& Shield}, \tequip{Heavy Weapons}, \tequip{Polearms}, \tequip{Claws / Gloves}, \tequip{Twin Weapons}, \tequip{Bows}, \tequip{Throwing Weapons}, \tequip{Rifle / Crossbows}, \tequip{Staffs}, \tequip{Wands}, and \tequip{Instruments}.

The character’s Abilities indicate which weapons and armor they may equip. If a character wishes to equip a weapon that they may not, it doesn’t add their Offensive Stat to any roll or attack; If a character wishes to equip an armor that they may not, they don’t consider their Defensive Stat in opponents’ rolls or attacks.

All equipment has a minimum level to be used, and most have a Gil cost. The items that do not have Gil costs are so rare that there are only a few copies of each in the world or artifacts so hard to be found that there are at most one, maybe two copies worldwide. Rare and artifact items should be given at the GM’s discretion.

\subsection{Armor}
\label{subsec:inv-armor}

Armors are the main defensive equipment. Although some specific armor can empower attacks, their main function is to reduce the damage suffered. All armor pieces have two characteristics: Armor (ARM) and Magic Armor (MARM). Every time you suffer physical damage, decrease the damage by an amount equal to your ARM. Every time you suffer magical damage, decrease the damage by an amount equal to your MARM.

\begin{boco}
    For example, a character with MARM 21 suffers a magical attack dealing 46 damage. They lose 25 HP due to their magical protection. Regardless of ARM and MARM values and the damage dealt, every successful attack that deals damage causes a minimal loss of 1 HP. As it takes some time to don an armor suit, it is impossible to equip or change armor in combat.
\end{boco}

\end{multicols}

\begin{center}
    \adjincludegraphics[width=0.75\textwidth]{block-entering-building}
\end{center}
\clearpage

\subsubsection{Light Armor}

This armor is commonly used by Mages and Rune Knightss. Its main purpose is to protect against magical damage, leaving it relatively ineffective against physical damage.

\begin{tabarm}[label=inv-larm]
    \tabarmrow[name=Cotton Robe,mlevel=1,cost=100,arm=0,marm=3,effect=None]
    \tabarmrow[name=Snow Cape,mlevel=1,cost=142,arm=0,marm=3,effect=Auto - \tstatus{Resist (Ice)}]
    \tabarmrow[name=Mistle Robe,mlevel=1,cost=143,arm=0,marm=3,effect=\tstatus{Sleep} Immunity]
    \tabarmrow[name=Leather Robe,mlevel=10,cost=585,arm=2,marm=10,effect=None]
    \tabarmrow[name=Temple Cloth,mlevel=10,cost=866,arm=2,marm=10,effect=Auto - \tstatus{Resist (Shadow)}]
    \tabarmrow[name=Robe of Lords,mlevel=64,cost=Artifact,arm=87,marm=179,effect=Add Twice Fire level to Spells' damage; Maximum MP + 20\%]
\end{tabarm}

\clearpage
\subsubsection{Medium Armor}

This armor type is common between many different Jobs.  It is balanced between physical and magic defenses, leaving no easily exploitable weakness.

\begin{tabarm}[label=inv-marm]
    \tabarmrow[name=Leather Outfit,mlevel=1,cost=84,arm=1,marm=1,effect=None]
\end{tabarm}

\clearpage
\subsubsection{Heavy Armor}

Favoring defense against physical damage, this type of armor is popular among Warriors and Adepts.  However, it does leave the wearer more vulnerable to magical damage.

\begin{tabarm}[label=inv-harm]
    \tabarmrow[name=Leather Plate,mlevel=1,cost=99,arm=3,marm=0,effect=None]
\end{tabarm}

\clearpage
\subsection{Weapons}
\label{subsec:inv-weapons}

\begin{multicols}{2}
Weapons are the main offensive equipment.  Unlike armor, which are very similar, each type of weapon is quite different from others.  Each type of weapon has an Offensive Stat and a Defensive Stat.  When you use the \taction{Attack} or any other action that specifies a ``weapon attack'', you attack using the equipped weapon's Offensive and Defensive Stats unless the action says otherwise.  For example, Flamberge's (a Light Sword / Knife) damage is equal to 10x.  As its Offensive Stat is Air, the damage will be equal to ten times your Air level before any other modifiers.

In addition, the attack and damage type (physical or magical), the element (usually Cut, Puncture, or Crush), and if the attack is Ranged or Melee, all depends on the equipped weapon.

Every action that needs a weapon attack to hit and does not state the action's range (Ranged or Melee) and/or the action's elements uses the base weapon's range and element.  Actions that do state their range and/or element override the base weapon's characteristics.

Many weapons also have special effects.  All effects that modify the weapon's attack or damage only work with the \taction{Attack} action, unless the action says otherwise.  The exceptions to this rule are all effects that increase damage by some Stat level like the Colichemarde and the Soul Eater weapon ability.  Actions that key off weapon damage use the damage increased by the special effect and the Soul Eater applies to all actions made with that weapon that require a weapon attack.
\end{multicols}

\begin{center}
    \adjincludegraphics[width=0.9\textwidth]{block-shopping}
\end{center}
\clearpage

\subsubsection{Light Swords / Knives}

Fast weapons that deal \telem{Puncture}-elemental damage.  Their Offensive and Deffensive Stats are Air.  Due to their weak damage, they are largely a defensive option.  While a character is equipped with one of these weapons, they may use the reaction \taction{Parry}.  It is used when they suffer a physical attack to make a roll Air vs Air at difficulty 40.  If successful, they don't suffer the attack's effects.  These weapons are always Melee.

\begin{tabwpn}[label=inv-lsword,range=melee,type=physical,element=puncture,roll=airvair]
    \tabwpnrow[name=Epee,mlevel=1,cost=55,damage=2x,effect=None]
\end{tabwpn}
